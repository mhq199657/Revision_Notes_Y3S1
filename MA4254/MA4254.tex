\PassOptionsToPackage{svgnames}{xcolor}
\documentclass[12pt]{article}



\usepackage[margin=1in]{geometry}  
\usepackage{graphicx}             
\usepackage{amsmath}              
\usepackage{amsfonts}              
\usepackage{framed}               
\usepackage{amssymb}
\usepackage{array}
\usepackage{amsthm}
\usepackage[nottoc]{tocbibind}
\usepackage{bm}
\usepackage{enumitem}


\DeclareMathOperator{\Tr}{Tr}
 \newcommand{\im}{\mathrm{i}}
  \newcommand{\diff}{\mathrm{d}}
  \newcommand{\col}{\mathrm{Col}}
  \newcommand{\row}{\mathrm{R}}
  \newcommand{\kerne}{\mathrm{Ker}}
  \newcommand{\nul}{\mathrm{Null}}
  \newcommand{\nullity}{\mathrm{nullity}}
  \newcommand{\rank}{\mathrm{rank}}
  \newcommand{\Hom}{\mathrm{Hom}}
  \newcommand{\id}{\mathrm{id}}
  \newcommand{\ima}{\mathrm{Im}}
  \newcommand{\lcm}{\mathrm{lcm}}
  \newcommand{\st}{\mathrm{s.t.}}
  \newcommand{\T}{\mathrm{T}}
  \newcommand{\va}{\mathbf{a}}
  \newcommand{\cone}{\mathrm{cone}}
  \newcommand{\conv}{\mathrm{conv}}
  \newcommand\norm[1]{\left\lVert#1\right\rVert}
\setlength{\parindent}{0cm}
\setlength{\parskip}{0em}
\newcommand{\Lim}[1]{\raisebox{0.5ex}{\scalebox{0.8}{$\displaystyle \lim_{#1}\;$}}}
\newtheorem{definition}{Definition}[section]
\newtheorem{theorem}{Theorem}[section]
\newtheorem{notation}{Notation}[section]
\theoremstyle{definition}
\DeclareMathOperator{\arcsec}{arcsec}
\DeclareMathOperator{\arccot}{arccot}
\DeclareMathOperator{\arccsc}{arccsc}
\DeclareMathOperator{\spn}{Span}
\DeclareMathOperator{\x}{\mathbf{x}}
\setcounter{tocdepth}{1}
\begin{document}

\title{Revision notes - MA4254}
\author{Ma Hongqiang}
\maketitle
\tableofcontents

\clearpage
%\twocolumn
\section{Graph and Digraph}
\subsection{Graphs}
\begin{definition}[Graph]
\hfill\\\normalfont A graph $G$ is a pair $(V,E)$, where
\begin{itemize}
  \item $V$ is a finite set, and
  \item $E$ is a set of unordered pairs of elements of $V$.
\end{itemize}
Elements of $V$ are called vertices and elements of $E$ edges.\\
A pair of distinct vertices are \textbf{adjacent} if they define an edge. The edge is said to be \textbf{incident} to its defining vertices.\\
The \textbf{degree} of a vertex $v$, denoted as $\deg(v)$, is the number of edges incident to that vertex.
\end{definition}
\begin{definition}[Path, Cycle]
\hfill\\\normalfont A $v_1v_k$-\textbf{path} is a sequence of edges
\[
\{v_1, v_2\}, \{v_2, v_3\},\ldots, \{v_{k-1}, v_k\}
\]
A \textbf{cycle} is a sequence of edges
\[
\{v_1, v_2\}, \{v_2, v_3\},\ldots, \{v_{k-1}, v_k\}, \{v_k, v_1\}
\]
In both case we require the vertices to be all \textit{distinct}.\\
A graph is said to be \textbf{acyclic} if it has no cycle.
\end{definition}
\begin{theorem}
\hfill\\\normalfont If every vertex of $G$ has degree at least two, then $G$ has a cycle.
\end{theorem}
\begin{definition}[Connected Graph]
\hfill\\\normalfont $G$ is connected if each pair of vertices is connected by a path.
\end{definition}
\begin{theorem}
\hfill\\\normalfont Let $G$ be a connected graph with a cycle $C$ and let $e$ be an edge of $C$. Then $G-e$ is connected.
\end{theorem}
\begin{definition}[Subgraph]
\hfill\\\normalfont $H$ is a \textbf{subgraph} of $G$ if $V(H)\subseteq V(G)$ and $E(H)\subseteq E(G)$. \\
It is a spanning subgraph if in addition $V(H) = V(G)$.
\end{definition}
\begin{definition}[Tree]
\hfill\\\normalfont A \textbf{tree} is a connected acyclic graph.
\end{definition}
\begin{theorem}
\hfill\\\normalfont If $T=G(V,E)$ is a tree, then $|E|= |V|-1$.
\end{theorem}
\begin{theorem}
\hfill\\\normalfont :et $G=(V,E)$ be a connected graph. Then $|E|\geq |V|-1$. Moreover, $G$ is a tree if equality holds.
\end{theorem}
\begin{definition}[Bipartite Graph]
\hfill\\\normalfont $G=(S,T,E)$ is a bipartite graph if any edge in $E$ has one vertex in $S$ and the other in $T$.
\end{definition}
\begin{definition}[Vertex-Edge Incidence Matrix]
\hfill\\\normalfont The \textbf{vertex-edge incidence matrix} of a graph $G=(V,E)$ is a matrix $A\in \{0,1\}^{|V|\times|E|}$ such that
\begin{itemize}
\item The rows correspond to the edges of $G$,
\item the columns correspond to the edges of $G$,
\item entry $A_{v,ij}$ for vertex $v$ and edge $ij$ is
\[
A_{v,ij} = \begin{cases}
0&\text{if }v\neq i\text{ and }v\neq j\\
1&\text{if }v=i\text{ or }j
\end{cases}
\] 
\end{itemize}
\end{definition}
\begin{definition}[Digraph]
\hfill\\\normalfont A \textbf{directed graph}(digraph) $D$ is a pair $(N,A)$ where
\begin{itemize}
  \item $N$ is a finite set and
  \item $A$ is a set of ordered pairs of elements of $N$
\end{itemize}
Elements of $N$ are called nodes and elements of $A$ arcs.
\begin{itemize}
  \item Node $i$ is the tail of arc $(i,j)$.
  \item Node $j$ is the head of arc $(i,j)$.
\end{itemize}
The in-degree (resp. out-degree) of node $v$, denoted $\deg^+(v)$(resp. $\deg^-(v)$) is the number of arcs with head(resp. tail) $v$.
\end{definition}
\begin{definition}[Bipartite Digraph]
\hfill\\\normalfont A bipartite digraph is defined as $D=(S,T,A)$ where for all edges in $A$, it is incident from a node in $S$ to a node in $S$.
\end{definition}
\begin{definition}[Node-Arc Incidence Matrix]
\hfill\\\normalfont The \textbf{node-arc incidence matrix} of a graph $D=(N,A)$ is a matrix $M\in\{0,\pm 1\} ^{|V|\times |A|}$ such that
\begin{itemize}
  \item The rows correspond to the nodes of $D$.
  \item The columns correspond to the arcs of $D$
  \item the entry $M_{v,ij}$ for node $v$ and arc $ij$ is
  \[
M_{v,ij} = \begin{cases}
0&\text{ if } v\neq i \text{ and } v\neq j\\
-1&\text{ if } v=j,\\
+1&\text{ if } v=i
\end{cases}
  \]
\end{itemize}
\end{definition}
\subsection{Convex Set}
\begin{definition}[Convex Set]
\hfill\\\normalfont A set $S\subseteq\mathbb{R}^n$ is \textbf{convex} if for any $x,y\in S$ and any $\lambda\in[0,1]$, we have $\lambda x+(1-\lambda)y\in S$
\end{definition}
\subsection{Hyperplanes and Half Spaces}
\begin{definition}[Hyperplane]
\hfill\\\normalfont Let $a\in\mathbb{R}^n\setminus\{0\}$ abd $b\in\mathbb{R}$. Then the set 
\[
\{x\in\mathbb{R}^n \mid a^\T x = b\}
\]
is called a \textbf{hyperplane}.
\end{definition}
Geometrically, the hyperplane above can be understood by expressing it in the form
\[
\{x\in\mathbb{R}^n\mid a^\T(x-x^0) = 0\}
\]
where $x^0$ is any point in the hyperplane, i.e., $a^\T x^0=b$.
\begin{definition}[Half Space]
\hfill\\\normalfont Let $a\in\mathbb{R}^n\setminus\{0\}$ and $b\in\mathbb{R}$. Then the set
\[
\{x\in\mathbb{R}^n\mid a^\T x\geq b\}
\]
is called a \textbf{halfspace}.
\end{definition}
Notably, a halfspace is a convex set.
\subsection{Polyhedra}
\begin{definition}[Polyhedron]
\hfill\\\normalfont A \textbf{polyhedron} is a set of the form $\{x\in\mathbb{R}^n\mid Ax\geq b\}$, for some $A\in\mathbb{R}^{m\times n}$ and $b\in\mathbb{R}^m$.
\end{definition}
By letting $A=\begin{pmatrix} a_1^\T\\\vdots\\ a_m^\T\end{pmatrix}$ and $b=\begin{pmatrix} b_1\\\vdots\\b_m\end{pmatrix}$, we can understand the polyhedron to be the intersection of the halfspaces
\[
\{x\in\mathbb{R}^n \mid a_i^\T x\geq b_i\}, \;\;\; i = 1,\ldots, m
\]
From this point of view, we can see that the intersection of two polyhedra is again a polyhedron.\\
A bounded polyhedron is sometimes called a \textbf{polytype}.
\begin{definition}[Convex Combination]
\hfill\\\normalfont Let $x^1,\ldots, x^k\in\mathbb{R}^n$ and let $\lambda_1,\ldots, \lambda_k$ be nonnegative scalars whose sum is one.
\begin{enumerate}
  \item The vector $\sum_{i=1}^n \lambda_ix^i$ is said to be a \textbf{convex combination} of the vectors $x^1,\ldots, x^k$.
  \item The \textbf{convex hull} of the vectors $x^1,\ldots, x^k$ is the set of all convex combinations of these vectors.
\end{enumerate}
\end{definition}
\subsection{Basic Feasible Solution}
\begin{definition}[Extreme Point]
\hfill\\\normalfont Letr $P\subseteq \mathbb{R}^n$ be a polyhedron. A vector $x\in P$ is an \textbf{extreme point} of $P$ if we cannot find $y,z\in P$ distinct from $x$, and $\lambda\in[0,1]$, such that $x=\lambda y+(1-\lambda)z$.
\end{definition}
\begin{definition}[Active Constraint]
\hfill\\\normalfont 
Consider a polyhedron $P\in\mathbb{R}^n$, partition the constraints according to the sign:
\begin{itemize}
  \item $a_i^\T x\geq b_i, i\in M_1$
  \item $a_i^\T x\leq b_i, i\in M_2$
  \item $a_i^\T x= b_i, i\in M_3$
\end{itemize}
where $M_1, M_2, M_3$ are finite index sets, each $a_i\in\mathbb{R}^n\setminus\{0\}$ and each $b_i\in \mathbb{R}$.\\
If $x^\ast\in\mathbb{R}^n$ satisfies $a_i^\T x^\ast = b_i$ for some $i\in M_1, M_2, M_3$, we say that the corresponding constraint is \textbf{active} at $x^\ast$. The \textbf{active set} of $P$ at $x^\ast$ is
\[
I(x^\ast) = \{i\in M_1\cup M_2\cup M_3\mid a_i^\T x^\ast = b_i\}
\]
i.e., $I(x^\ast)$ is the set of indices of constraints active at $x^\ast$.
\end{definition}
\begin{theorem}[Linear Algebra Equivalence]
\hfill\\\normalfont Let $x^\ast \in \mathbb{R}^n$. The following are equivalent.
\begin{enumerate}
	\item There are $n$ linearly independent veectors in the set $\{a_i\mid i\in I(x^\ast)\}$.
	\item The span of the vectors $a_i, i\in I(x^\ast)$, is all of $\mathbb{R}^n$.
	\item The system $a_i^\T x=b_i, i\in I(x^\ast)$, has a unique solution.
\end{enumerate}
\end{theorem}
\begin{definition}[Basic Solution, Basic Feasible Solution]
\hfill\\\normalfont The vector $x^\ast\in\mathbb{R}^n$ is called a \textbf{basic solution} if
\begin{enumerate}
	\item $a_i^\T x^\ast = b_i, i\in M_3$
	\item There are $n$ linearly independent vectors in $\{a_i\}_{i\in I(x^\ast)}$.
\end{enumerate}
A \textbf{basic feasible solution}(BFS) is a basic solution satisfying all constraints, namely in $M_1, M_2, M_3$.
\end{definition}
\subsection{Finite Basis Theorem for Polyhedra}
\begin{definition}[Cone]
\hfill\\\normalfont A set $C\subseteq \mathbb{R}^n$ is a \textbf{cone} if $\lambda x\in C$ for all $\lambda\geq 0$ and $x\in C$.
\end{definition}
It is obvious from definition that $0\in C$.
\begin{definition}[Convex Cone]
\hfill\\\normalfont For vectors $x^1,\ldots, x^k\in \mathbb{R}^n$, let
\[
\cone\{x^1,\ldots, x^k\} := \{x\in\mathbb{R}^n \mid x=\sum_{i=1}^k \lambda_i x^i, \lambda_i\geq 0, i=1,\ldots, k\}
\]
Then, $\cone\{x^1,\ldots, x^k\}$ is a cone and convex set, which is called the \textbf{convex cone} generated by $x^1,\ldots, x^k$.
\end{definition}
\begin{definition}[Polyhedral Cone]
\hfill\\\normalfont The set $P=\{x\in \mathbb{R}^n\mid Ax\geq 0\}$ is called a \textbf{polyhedral cone}.
\end{definition}
By a theorem of Weyl, we have $\cone\{x^1,\ldots, x^k\}$ is a polyhedral cone.
\begin{definition}[Recession Cone]
\hfill\\\normalfont Given $A\in\mathbb{R}^{m\times n}$ and $b\in\mathbb{R}^m$, consider
\[
P=\{x\in\mathbb{R}^n\mid Ax\geq b\}
\] 
and $y\in P$.\\
The \textbf{recession cone} $R(P,y)$ of $P$ at $y$ is the set
\[
\{d\in\mathbb{R}^n \mid A(y+\lambda d)\geq b, \text{ for all } \lambda\geq 0\}
\]
It is easy to see the recession cone of $P$ at any $y$ is
\[
\{d\in\mathbb{R}^n\mid Ad\geq 0\}
\]
and is a polyhedral cone. Thus, the recession cone is independent of the starting point $y$, so we can denote the recession cone as $R(P)$.
\end{definition}
For $P=\{x\in\mathbb{R}^n \mid Ax=b,x\geq 0\}$, where $A\in \mathbb{R}^{m\times n}$ and $b\in\mathbb{R}^m$, the recession cone is
\[
\{d\in\mathbb{R}^n\mid Ad=0, d\geq 0\}
\]
\begin{definition}[Extreme Rays]
\hfill\\\normalfont \textbf{Extreme rays} of a polyhedral cone $C\subseteq \mathbb{R}^n$ are elements $d\neq 0$ such that there are $n-1$ linearly independent constraints active at $d$.\\
\textbf{Extreme rays} of a nonempty polyhedron $P$ are the extreme rays of the recession cone of $P$.
\end{definition}
\begin{definition}[Minkowski Sum]
\hfill\\\normalfont Let $A,B\subseteq \mathbb{R}^n$. The Minkowski sum $A+B$ is
\[
A+B:=\{a+b: a\in A, b\in B\}
\]
\end{definition}
\begin{theorem}[Finite Basis Theorem for Polyhedra]
\hfill\\\normalfont Let $P=\{x\in\mathbb{R}^n\mid Ax\leq b\}$, where $\rank(A) = n$, $A\in\mathbb{R}^{m\times n}$. Let $\{x^1,\ldots, x^q\}$ be the set of extreme points of $P$, which $P$ has finitely many of, and let $\{d^1,\ldots, d^r\}$ be the set of extreme rays of $P$. Then,
\[
P=\conv\{x^1,\ldots, x^q\}+\cone\{d^1,\ldots, d^r\}
\]
\end{theorem}
Therefore, it can be shown that $P$ is \textbf{bounded} \textit{if and only if} the recession cone of $P$ contains zero vector only.
\begin{theorem}[Farkas' Lemma]
\hfill\\\normalfont This lemma can be used, together with LP duality to prove the Finite Basis Theorem. The lemma goes:\\
Exactly one of the following holds:
\begin{enumerate}
	\item $\{x\in\mathbb{R}^n\mid Ax=b,x\geq 0\}\neq \varnothing$ or
	\item $\{y\in\mathbb{R}^m\mid A^\T y\geq 0, y^\T b<0\}\neq \varnothing$.
\end{enumerate}
\end{theorem}






\end{document}
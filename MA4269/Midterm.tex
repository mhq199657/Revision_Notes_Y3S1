\PassOptionsToPackage{svgnames}{xcolor}
\documentclass[12pt]{article}



\usepackage[margin=0.1in]{geometry}  
\usepackage{graphicx}             
\usepackage{amsmath}              
\usepackage{amsfonts}              
\usepackage{framed}               
\usepackage{amssymb} 
\usepackage{array}
\usepackage{amsthm}
\usepackage[nottoc]{tocbibind}
\usepackage{bm}
\usepackage{enumitem}


  \newcommand\norm[1]{\left\lVert#1\right\rVert}
\setlength{\parindent}{0cm}
\setlength{\parskip}{0em}
\newcommand{\Lim}[1]{\raisebox{0.5ex}{\scalebox{0.8}{$\displaystyle \lim_{#1}\;$}}}
\newtheorem{definition}{Definition}[section]
\newtheorem{theorem}{Theorem}[section]
\newtheorem{notation}{Notation}[section]
\theoremstyle{definition}
\DeclareMathOperator{\arcsec}{arcsec}
\DeclareMathOperator{\arccot}{arccot}
\DeclareMathOperator{\arccsc}{arccsc}
\DeclareMathOperator{\PV}{PV}
\DeclareMathOperator{\TV}{TV}
\DeclareMathOperator{\diff}{d}
\DeclareMathOperator{\expec}{E}
\DeclareMathOperator{\var}{Var}
\DeclareMathOperator{\cov}{Cov}
\DeclareMathOperator{\CE}{CE}
\DeclareMathOperator{\RP}{RP}
\newcommand\cf[1]{\mathbf{#1}}
\setcounter{tocdepth}{1}
\setcounter{section}{-1}
\begin{document}



%\clearpage
\twocolumn
\section{Review}
\subsection{Introduction}
\begin{definition}[Zero-coupon Bond]
\normalfont A contract to deliver $\$1$ on a future date $T$ is known as a \textbf{zero-coupon bond}.\\
The price of the bond at time $t<T$ is denoted by $P(t,T)$, where $T$ is the maturity of the bond.
\end{definition}
\begin{definition}[Money Market Account]
\normalfont The \textbf{money market account} is an asset created by the following procedure, called continuously compounded:
\begin{itemize}
  \item The initial amount equal to $\$1$ is invested at time $t=0$ in the bond with the shortest available maturity(i.e. the next infinitesimal instant).
  \item The position is rolled over to the bond with the next shortest maturity once the first bond expires.
\end{itemize}
The price of the money market at time $t$ is denoted by $M_t$.
\end{definition}
\begin{theorem}
\normalfont We have
\[
\diff M_t = r(t)M_t\diff t
\]
to describe the value of money market account, where $r(t)$ is a parameter known as the interest rate.\\
Solving, we have
\[
M_t = \exp\left(\int_0^t r(u)\diff u\right)
\]
\end{theorem}
\textbf{Remark}: The relationship between $M_t$ and $P(t,T)$ is non-trivial if $r(t)$ is stochastic. However, 
\begin{theorem}
\normalfont Suppose interest rate $r$ is constant. Then, 
\[
P(t,T)=\frac{M_t}{M_T} = e^r(T-t)
\]
It is obvious that $P(t,T)<1$ as long as $r>0$.
\end{theorem}
\begin{definition}[Value of Asset]
\normalfont The \textbf{value} of an asset is the amount of dollars that an investor will pay to own that asset.\\
The term \textbf{stock price} refers to the value (per unit) of the stock under consideration.
\end{definition}
\begin{definition}[Position]
\normalfont A \textbf{position} in an asset is the equantity of an asset owned or owed by an investor.
\begin{itemize}
  \item \textbf{long position}: the investor \textit{owns} the asset.
  \item \textbf{short position}: the investor \textit{sells} the asset that he does not own.
\end{itemize}
\end{definition}
\begin{definition}[Portfolio]
\normalfont A \textbf{portfolio} is a combination of various positions in financial assets.\\
At any time $t$, the \textbf{value of the portfolio} $\Pi_t$ is just eh sum of the values of all the positions held in the portfolio at that particular time $t$:
\[
\Pi_t = a_t^{(1)}A_t^{(1)}+\cdots +a_t^{(n)}A_t^{(n)}
\]
where $a_t^{(i)}$ is the position where $A_t^{(i)}$ is the price of the asset at time $t$.\\
Since the price of assets is not determined by us, therefore, $\Pi_t$ can be written as the vector 
\[
(a_t^{(1)}),\ldots, a_t^{(n)})
\]
\end{definition}
\begin{definition}[Self-financing]
\normalfont A portfolio $\Pi_t=(a_t^{(1)}),\ldots, a_t^{(n)})$ is \textbf{self-financing} over the time interval $[0,T]$ if there is \textit{no} exogeneous infusion or withdrawal of money, \textit{except} possibly at the initiation time $0$ or maturity date $T$:
\[
\diff\Pi_t = a_t^{(1)}dA_t^{(1)}+\cdots +a_t^{(n)}dA_t^{(n)}
\]
Roughly speaking, the differential equation says tha thte change in portfolio is completely due to the change in the underlying asset prices and nothing else.
\end{definition}
\begin{definition}[Direction of Cash Flow]
\normalfont Suppose I am an investor
\begin{itemize}
  \item If cash flow is positive, it measn that someone pays me.
  \item If cash flow is negative, it means that I pay someone.
\end{itemize}
\end{definition}
Cash flow depends on
\begin{enumerate}
  \item the value of the underlying portfolio, and
  \item whether or not the portfolio is entered into or liquidated.
\end{enumerate}
\subsection{Financial Market}
\begin{definition}[Arbitrage]
\normalfont An \textbf{arbitrage opportunity} is the existence of a self-financing portfolio $\Pi_t, 0\leq t\leq T$, having the following properties:
\begin{enumerate}
  \item $\Pi_0=0$
  \item $\Pi_T\geq 0$ for \textbf{all possible} outcomes
  \item There is a positive probability that $\Pi_T>0$.
\end{enumerate}
\end{definition}
\begin{definition}[Equivalent Definition of Arbitrage]
\normalfont An arbitrage opportunity is the existence of a self-financing portfolio $\Pi_t, 0\leq t\leq T$, having the following properties:
\begin{enumerate}
  \item $\Pi_T-\Pi_0e^{rT}\geq 0$ for \textbf{all possible} outcomes
  \item There is a positive probability that $\Pi_T-\Pi_0e^{rT}>0$
\end{enumerate}
\end{definition}
\begin{theorem}[Consequence of No Arbitrage]
\normalfont Suppose there are two self-financing portfolios $\Pi_t^A$ and $\Pi_t^B$ over the time interval $[t,T]$ such that $\Pi_T^A\geq \Pi_T^B$. Then in the absence of arbitrage, we must have
\[
\Pi_t^A\geq \Pi_t^B
\]
\end{theorem}
\begin{theorem}[Law of One Price]
\normalfont Law of one price is a consequence of the above theorem.\\
Suppose there are two self-financing portfolios $\Pi_t^A$ and $\Pi_t^B$ over the time interval $[t,T]$ such that $\Pi_T^A = \Pi_T^B$. Then, in the absence of arbitrage, we must have
\[
\Pi_t^A = \Pi_t^B
\]
\end{theorem}
\begin{theorem}
\normalfont All \textit{risk-free} portfolios must earn the same return, i.e., \textit{riskless interest rate}. Suppose $\Pi_t$ is teh valueo f a riskfree portfolio, and $\diff\Pi_t$ is the price increment during a small period of time interval $[t,t+\diff t]$. Then
\[
\diff \Pi_t = r\Pi_t\diff t
\]
where $r$ is the riskless interest rate.
\end{theorem}
\subsection{Forward Contracts \& Options}
\begin{definition}[Forward Contract]
\normalfont A \textbf{forward contract} is a contract that delivers one unit of the underlying asset on a known future date $T$ for a certain price $K$ agreed today.\\
Here,
\begin{itemize}
  \item $K$ is the \textbf{delivery price}
  \item $T$ is called the \textbf{delivery date}
  \item the buyer of the contract is in the long position
  \item the seller of the contract is in the short position
  \item the delivery price $K$ is the amount the long sides pays the short side in exchange of one unit of the \textbf{udnerlying asset whose value} is $S_T$ on the delivery date $T$.
\end{itemize}
\end{definition}
\begin{definition}[Forward Price]
\normalfont The \textbf{forward price} at time $t$ is the delivery price of a forward contract which costs nothing to enter into at time $t$.
We denote the forward price at time $t$ by
\[
F(S_t, t, T)
\]
\end{definition}
\textbf{Remark}: The forward price $F(S,t,T)$ is \textit{not} the value of corresponding forward contract.
\begin{definition}[Payoff, Profit]
\normalfont The \textbf{payoff} to a position is the value of the position at the maturity date $T$.\\
The \textbf{profit} to a position is the payoff to the position at maturity dat $T$, subtracted by the time-$T$ value of the initial investment in the position:
\[
\Pi_T-\Pi_te^{r(T-t)}
\]
\end{definition}
\begin{theorem}[Payoff of Forward Contract]
\normalfont It is obvious from the definition that 
\begin{itemize}
  \item the payoff to a long forward contract is $S_T-K$;
  \item the payoff to a short forward contract is $K-S_T$
\end{itemize}
\end{theorem}
Suppose it costs nothing to enter into a forward contract, then by using the forward price definition, 
\begin{itemize}
  \item the payoff and the profit to a long forward contract are the same:
  \[
S_T-F(S, t, T)
  \]
  \item the payoff and profit to a short forward contract are the same:
  \[
F(S,t,T) - S_T
  \]
\end{itemize}
\begin{theorem}[Forward Price]
\normalfont Suppose the underlying stock $S$ does not pay dividends. Then the forward price $F(S,t, T)$ of stock at time $t$ is given by
\[
F(S,t,T) = Se^{r(T-t)}
\]
where $S$ is the price of the stock at time $t$.
\end{theorem}
\begin{definition}[Call Option]
\normalfont A \textbf{call option} is an agreement where the buyer has the \textit{right}, but not the obligation to buy the underlying asset, for a certain price $K$ agreed at the initiation of the contract. Here, $K$ is called the stike price, whereby $T$ is used to denote maturity, which is the darte by which option must be exercised or it becomes worthless. \\
For now, we consider only European call option, where exercise of the contract occurs only at maturity $T$.\\
The payoff to a long European call option with strike price $K$ and maturity $T$ is
\[
(S_T-K)^+=\max\{S_T-K,0\}
\]
where the payoff to a short European call option with same strike price and maturity is $-(S_T-K)^+$.
\end{definition}
\begin{definition}[Put Option]
\normalfont A \textbf{put option} is an agreement where the buyer has the right to sell an asset, but not the obligation to sell, for a certain price $K$ agreed at the initiation of the contract.\\
The payoff to a long European put option with strike price $K$ and expiration $T$ is
\[
(K-S_T)^+=\max\{K-S_T,0\}
\]
whereas the payoff to a short European put option with same strike price and expiration $T$ is $-(K-S_T)^+$.
\end{definition}
\begin{definition}[Moneyness]
\normalfont Options are often described by their degree of moneyness. At any time $t$, an option is said to be 
\begin{itemize}
  \item \textbf{in-the-money} if payoff at time $t>0$.
  \item \textbf{at-the-money} if payoff$=0$, i.e., $S_t=K$.
  \item \textbf{out-of-the-money} if payoff$<0$.
\end{itemize}
\end{definition}
\begin{theorem}[Put Call Parity]
\normalfont We have the following relationship between call $c$ and put $p$ price, over the underlying asset at time $t$. Here $K$ is the strike price of the options and $F$ is the forward price at time $t$.
\[
c-p+(K-F)e^{-r(T-t)}=0
\]
\end{theorem}
\subsection{Binomial Model}
\begin{definition}[One-period Binomial Model]
\normalfont Suppose the non-dividend paying stock price per share today is $S_0$. We assume that, at the end of the one period, the stock price is either $S_0u$ or $S_0d$ where $d$ and $u$ are positive real numbers such that $d<u$.\\
We call $u$ the \textbf{up factor} and $d$ the \textbf{down factor}.\\
Consider a derivative on the stock with time $T$ to maturity. Let $V_0$ be the price of derivative at time $0$.\\
We can price $V_0$ by constructing $\Pi_0=V_0-\phi S_0$, and make it riskless, i.e. $\Pi_T = V_u-\phi S_0u=V_d-\phi S_0d$, by picking a suitable $\phi$. Since $\Pi_0$ is riskless, its payoff should be the same as any other riskless payoff, e.g. money market account,i.e., $\Pi_T = (V_0-\phi S_0)e^{rT}$.\\
Solving, we have
\[
V_0 = e^{-rT}(pV_u+(1-p)V_d)
\]
where $p=\frac{e^{rT}-d}{u-d}$.
\end{definition}
We can interpret $p$ and $1-p$ as probabilities distribution on $S_T$, so that we can write
\[
V_0 = e^{-rT}E^\mathbb{Q}[V_T]
\]
The expectation of $S_T$ under $\mathbb{Q}$ is
\[
E^\mathbb{Q}[S_T] = S_0e^{rT}
\]
which matches our riskless argument.
\begin{theorem}[Restriction on {$u$} and {$d$}]
\normalfont In the one-period binomial model where the one period is $[0,T]$ and the corresponding up-factor and down-factor of a non-dividend paying stock are $u$ and $d$ respectively with $d<u$, we have
\[
d<e^{rT}<u
\]
\end{theorem}
\begin{definition}[Multi-period Binomial Model]
\normalfont At any time $j\Delta t$, there are $j+1$ possible stock prices:
\[
S_0d^j, S_0d^{j-1}u,\ldots, S_0u^j
\]
Without loss of generality, we can assume that $ud=1$.\\
If $V_j^k$ is the price of the derivative at time $j\Delta t$ when the underlying stock price is $S_0d^{j-k}u^k$, i.e. there are $k$ period out of $j$ that the price goes up.\\
We then have
\[
V_0 = e^{-rn\Delta t}\sum_{i=0}^n\binom{n}{i}p^i(1-p)^{n-i}V_n^i
\]
where $V_0$ is the price of the European style derivative given by the $n$ period binomial model.
\end{definition}
%\clearpage
\section{Brownian Motion}
\subsection{Brownian Motion}
\begin{definition}[Standard Brownian Motion]
\normalfont A \textbf{standard Brownian Motion} is a stochastic process $W_t, t\geq 0$ with the following defining characteristics:
\begin{itemize}
  \item[(W1)] $W_0=0$.
  \item[(W2)] With probability $1$ (almost surely), the function $t\to W_t$ is continuous in $t$.
  \item[(W3)] For every $0\leq t_1<t_2$, $W_{t_2}-W_{t_1}$ is normally distributed with mean $0$ and variance $t_2-t_1$.
  \item[(W4)] $W_{t_3}-W_{t_2}$ is independent of $W_{t_1}-W_{t_0}$ for any $0\leq t_0\leq t_1\leq t_2\leq t_3$, i.e. non-overlapping increments are independently distributed.
\end{itemize}
\end{definition}
Property (W3) implies that for any $\Delta t$, $W_{t+\Delta t}-W_t\sim N(0,\Delta t)$. This implies that $|W_t|\leq 1.96\sqrt{t}$ with 95\% probability.\\
Property (W4) implies that $\cov (W_t, W_s) = E[W_tW_s] = \min\{t,s\}$.\footnote{We write $W_t=(W_t-W_s+W_s)$ if $t>s$.}
\begin{theorem}[Binomial Approximation to Brownian Motion]
\normalfont Let $\varepsilon_1,\ldots$ be a sequence of independent, identically distributed random variables with mean $0$ and variance $1$. For each $n\geq 1$, define a continuous time stochastic process $W_t^{(n)}$ by
\[
W_t^{(n)} = \frac{1}{\sqrt{n}} \sum_{1\leq i\leq [nt]} \varepsilon_i
\]
$W_t^{(n)}$ approaches a standard Brownian motion $N(0,t)$ as $n\to \infty$.
\end{theorem}
\begin{theorem}[Infinitesimal Brownian Increment]
\normalfont We can approximate $\diff W_t$ is a very small time interval $\Delta t$:
\[
\Delta W_t:=W_{t+\Delta t} - W_t
\]
and then we have
\[
\Delta W_t = \phi \sqrt{\Delta t} \text{i.e., }\Delta W_t \sim N(0, \Delta t)
\]
where $\phi$ has a standard normal distribution.
\end{theorem}
\begin{definition}[Generalised Wiener Process]
\normalfont A \textbf{generalised Wiener Process} for a variable $X$ can be defined in terms of $\diff W_t$ as
\[
\diff X_t = a\diff t + b\diff W_t
\]
where $a$ and $b$ are constants. The parameter $a$ is called the \textbf{drift rate} and $b^2$ is called the \textbf{variance rate} of the process.\\
In a small time interval $\Delta t$, the change $\Delta X_t$ is given by
\[
\Delta X_t = a\Delta t + b\Delta W_t
\]
Therefore, $\Delta X_t$ has a normal distribution with mean $a\Delta t$ and variance $b^2\Delta t$.
\end{definition}
Here, we can safely write $X_t = X_0+at+bW_t$.
\subsection{Quadratic Variation}
\begin{definition}[Quadratic Variation]
Any sequence of values $0=t_0<t_1<\cdots<t_n=T$ is called a partition $\Pi=\Pi(t_0,\ldots, t_n)$ of a fixed interval $[0,T]$. The discrete quadratic variation of a standard Brownian motion $W$ relative to the partition $\Pi$ is defined as
\[
Q(W,\Pi) = \sum_{i=1}^n (W_{t_i}-W_{t_{i-1}})^2
\]
For any partition $\Pi$, define
\[
\|\Pi\| = \max_{1\leq i\leq n}|t_i-t_{i-1}|
\]
\end{definition}
\begin{theorem}[{$n$}th Moment of Standard Normal {$Z$}]
\normalfont The $n$th moment of a random variable $X$ is defined to be $E[X^n]$. If $\phi\sim N(0,1)$, then
\[
E[\phi^n] = \begin{cases}
0&\text{ if }n\text{ is odd}\\
\frac{(2k)!}{2^kk!}&\text{ if }n=2k
\end{cases}
\]
In particular, we have $E(\phi^2) = 1$ and $E(\phi^4) = 3$.
\end{theorem}
\begin{theorem}
\normalfont Consider an arbitrary sequence of partitions $\Pi_n$, where $n = 1,2,\ldots$. Suppose $\lim_{n\to \infty} \|\Pi_n\|=0$, then
\[
\lim_{n\to \infty} E(Q(W,\Pi_n)-T)^2] = 0
\]
That is the standard Brownian motion has quadratic variation which is equal to $T$, in the mean square limit.
\end{theorem}
\begin{definition}
\normalfont Define the integral $\int_0^T (\diff W)^2$ by
\[
\lim_{n\to infty} E[\sum_{i=1}^n (W_{t_i}-W_{t_{i-1}})^2 - \int_0^T(\diff W_t)^2]^2 = 0
\]
\end{definition}
However, from quadrativ variation theorem, we have
\[
\lim_{n\to infty} E[\sum_{i=1}^n (W_{t_i}-W_{t_{i-1}})^2 - \int_0^T\diff t]^2 = 0
\]
Therefore,
\[
\int_0^T(\diff W_t)^2=\int_0^T\diff t
\]
In fact, we can write
\[
(\diff W_t)^2 = \diff t
\]
which gives
\[
(\Delta W)^2 \approx \Delta t
\]
in discrete time approximation.
\subsection{It\^{o}'s lemma}
\begin{definition}[It\^{o}'s Process]
\normalfont The It\^{o}'s Process $\diff X_t$ is defined as 
\[
\diff X_t = a(X_t,t)\diff t + b(X_t,t)\diff W_t
\]
\end{definition}
\begin{theorem}[It\^{o}'s Lemma]
\normalfont Let $V(X_t,t)$ be a smooth function of $t$ and of the It\^{o}'s process $X_t$:
\[
\diff X_t = a\diff t + b\diff W_t
\]
for some $a=a(X_t,t)$ and $b=b(X_t,t)$. Then we have
\[
\diff V(X_t,t) = (a\frac{\partial V}{\partial X}+\frac{\partial V}{\partial t}+\frac{1}{2}b^2\frac{\partial^2 V}{\partial X^2})\diff t+b\frac{\partial V}{\partial X}\diff W_t
\]
\end{theorem}
Another version of the Ito's Lemma where we do not have explicit form of $\diff X$ is
\[
\diff V_t = \diff V(X_t,t) = \frac{\partial V}{\partial t}\diff t+\frac{\partial V}{\partial X}\diff X_t + \frac{1}{2}\frac{\partial^2 V}{\partial X^2}(\diff X_t)^2 
\]
\begin{definition}[Ito Integral]
\normalfont In order to have differential form of the SDE $\diff X_t = a(X_t,t)\diff t+b(X_t,t)\diff W_t$, we require $a(X_t,t)$ and $b(X_t,t)$ to be non-anticipative, which means that its value at $t$ can only be available at time $t$.\\
With the above assumption, we can define
\[
\int_0^T a(X_t,t)\diff t = \lim_{n\to \infty}\sum_{i=1}^n a(X_{t_{i-1}}, t_{i-1})(t_i-t_{i-1})
\]
and Ito integral
\[
\int_0^T b(X_t,t)\diff W_t
\]
is the mean-square limit of the sum $\sum_{i=1}^n b(X_{t_{i-1}}, t_{i-1})(W_{t_i}-W_{t_{i-1}})$.
\end{definition}
%\clearpage
\section{Black-Scholes Model}
In Black-Scholes Model, we assume the following 2 conditions:
\begin{enumerate}
  \item The money market(riskless asset) $M_t$ is given by
  \[
\diff M_t = rM_t\diff t
  \]
  \item The stock price follows the Geometric Brownian motion:
  \[
\diff S_t = \mu S_t\diff t +\sigma S_t\diff W_t
  \]
  where $\mu$ and $\sigma$ are constants.\\
  The explicit formula is $S_t=\exp((\mu-\frac{1}{2}\sigma^2)t+\sigma W_t)$.
\end{enumerate}
We can derive Black Scholes PED from either delta hedging, where we take $\Pi_t=V_t-\phi_tS_t, \;0\leq t\leq T$, such that $\phi_t$ is chosen to make $\Pi_t$ self-financing and riskless.\\
Self financing condition gives
\[
\diff \Pi_t=\diff V_t - \phi_t\diff S_t
\]
Also, since $\phi_t$ is chosen so that $\Pi_t$ is riskless, we also need
\[
\diff \Pi_t = r\Pi_t\diff t
\]
Here, $\diff V_t$ can be calculated via Ito's lemma and $\diff S_t$ is given in assumption. Solving, we will have
\[
\phi_t = \frac{\partial V}{\partial S}
\]
and
\[
\frac{\partial V}{\partial t}+r\frac{\partial V}{\partial S}S_t +\frac{1}{2}\sigma^2S_t^2\frac{\partial^2 V}{\partial S^2}=rV
\]
Here, we call $\frac{\partial V}{\partial S}$ \textbf{delta} of the derivative.\\
We can derive the Black Scholes PDE by \textit{replication} also, by considering an asset $\Pi_t=a_tS_t+b_tM_t$, which satisfies $\Pi_t=V_t$ for all $t\leq T$. The equality throughout $T$ gives $\diff \Pi_t=\diff V_t$.\\
Similarly, self financing condition gives
\[
\diff \Pi_t=a_t\diff S_t+v_t\diff M_t
\]
where $\diff S_t$ and $\diff M_t$ are readily available.\\
Also, we can conpute $\diff V_t$ via Ito's Lemma:
\[
\diff V_t = (\frac{\partial V}{\partial S}\mu S_t+\frac{\partial V}{\partial t}+\frac{1}{2}\frac{\partial^2V}{\partial S^2}\sigma^2S_t^2)\diff t+\frac{\partial V}{\partial S}\sigma S_t\diff W_t
\]
Then we compare coefficients of $\diff W_t$, arriving at
\[
a_t=\frac{\partial V}{\partial S}
\]
and therefore $b_t$ can be wriiten as
\[
b_t=\frac{1}{M_t}(V_t-\frac{\partial V}{\partial S}S_t)
\]
whereas comparing $\diff t$ and make necessary computation, we can arrive at
\[
\frac{\partial V}{\partial t}+r\frac{\partial V}{\partial S}S_t +\frac{1}{2}\sigma^2S_t^2\frac{\partial^2 V}{\partial S^2}-rV=0
\]
\textbf{Remark}:
\begin{enumerate}
  \item The drift parameter $\mu$ of the stock never enters into the PDE.
  \item To uniquely determine the solution, we must prescribe
  \begin{itemize}
    \item Boundary conditions
    \item Initial, or final conditions
  \end{itemize}
\end{enumerate}
\begin{theorem}[Solution to Black Scholes PDE]
\normalfont For a European call option with a call price $c(S_t,t)$, we can make the following observation:
\begin{enumerate}
  \item Final condition: $c(S_T,T)=\max\{S_T-K, 0\}$
  \item Boundary condition 1: $S_0=0\Rightarrow c(0,t)=0\text{ for all }0\leq t\leq T$.
  \item Boundary condition 2: With $S_t\gg K$, we have $c(S_t,t)\approx S_t$ for all $0\leq t\leq T$.
\end{enumerate}
With these observation, we have, for European call
\[
c_t= S_tN(d_{+})-Ke^{-r\tau}N(d_{-})
\]
For European put:
\[
p_t=Ke^{-r\tau}N(-d_{-})-S_tN(-d_+)
\]
where 
\[
d_{\pm}=\frac{\ln(S_t/K)+(r\pm \sigma^2/2)\tau}{\sigma\sqrt{\tau}}
\]
and
\[
\tau = T-t
\]
\end{theorem}
\begin{theorem}[Black Scholes PDE with Presence of Dividends]
\normalfont With presence of dividends,
\[
\diff \Pi_t = a_t\diff S_t+b_t\diff M_t+a_tqS_t\diff t
\] 
The black scholes PDE becomes
\[
\frac{\partial V}{\partial t}+(r-q)\frac{\partial V}{\partial S}S_t +\frac{1}{2}\sigma^2S_t^2\frac{\partial^2 V}{\partial S^2}-rV=0
\]
\end{theorem}
\subsection{Preliminaries on Martingale Pricing}
\begin{definition}[Equivalent Probability Measure]
\normalfont Suppose there are two probability measures $\mathbb{P}$ and $\mathbb{Q}$ on space $(\Omega, \mathcal{F})$. We say that $\mathbb{P}$ and $\mathbb{Q}$ are \textbf{equivalent}, denoted by $\mathbb{P}\sim \mathbb{Q}$ if
\[
\mathbb{P}(A)>0\Leftrightarrow \mathbb{Q}(A)>0\text{  for all }A\in\mathcal{F}
\]
\end{definition}
Essentially, two equivalent measures agree on \textbf{all} certain and impossible events.\\
We hope to derive an equivalent probability measure $\mathbb{Q}$ from an existing one $\mathbb{P}$. To do this, we require a \textbf{positive} random variable $L$ with property $\expec^{\mathbb{P}}[L]=1$. These two conditions are two defining characteristic of a Radon-Nikodym Derivative $L$.\\
Define $\mathbb{Q}$ by
\[
\mathbb{Q}(A)=\expec^{\mathbb{P}}[L\cdot \mathbf{1}_A]
\]
where $\mathbb{1}_A$ is the indicator random variable for the event $A$.
\begin{theorem}[Radon Nikodym]
\normalfont Consider two probability measures $\mathbb{P}$ and $\mathbb{Q}$ on $(\Omega, \mathcal{F})$. The following are equivalent:
\begin{enumerate}
  \item $\mathbb{P}\sim\mathbb{Q}$
  \item There eixsts a positive random variable $L$ such that for every event $A\in\mathcal{F}$
  \begin{itemize}
    \item $\mathbb{Q}(A)=\expec^\mathbb{P}[L\cdot \mathbf{1}_A]$
    \item $\mathbb{P}(A)=\expec^\mathbb{Q}[\frac{1}{L}\cdot \mathbf{1}_A]$.
  \end{itemize}
\end{enumerate}
Here $L=\frac{\diff \mathbb{Q}}{\diff \mathbb{P}}$, as derived from theorem, is called the Radon Nikodym derivative of $\mathbb{Q}$ wrt $\mathbb{P}$.
\end{theorem}
\begin{theorem}
\normalfont Let $X$ be a random variable. With above notations, we have
\[
\expec^\mathbb{Q}[X]=\expec^\mathbb{P}[L\cdot X]
\]
and
\[
\expec^\mathbb{P}[X]=\expec^\mathbb{Q}[\frac{1}{L}\cdot X]
\]
\end{theorem}
\begin{definition}[Filtration]
\normalfont Let $\{\mathcal{F}_t\}, 0\leq t\leq T$ be a filtration, where $\mathcal{F}_t$ is information available to us at time $t$.\\
Trivially, we have $\mathcal{F}_s\subseteq \mathcal{F}_t$ for $s\leq t$.
\end{definition}
Suppose we now want to move from $(\Omega, \{\mathcal{F}_t\}, \mathbb{P})$ to $(\Omega, \{\mathcal{F}_t\}, \mathbb{Q})$, we require a random variable $L_T$ satisfying the following:
\begin{itemize}
  \item $L_T$ is $\mathcal{F}_T$ measurable, which means that $L_T$ will be known at time $T$.
  \item $L_T$ is positive.
  \item $\expec^\mathbb{P}[L_T]=1$.
\end{itemize}
Define a stochastic process $L_t$ as follows:
\[
L_t=\expec^\mathbb{P}[L_T\mid \mathcal{F}_t], 0\leq t\leq T
\]
The above process is called \textbf{Radon-Nikodym} derivative/likelihood process.
\begin{theorem}
\normalfont Suppose $X_t,0\leq t\leq T$, is an adapted process on $\Omega, \mathcal{F}_t$. We have
\[
\expec^\mathbb{Q}[X_t]=\expec^\mathbb{P}[L_t\cdot X_t], 0\leq t\leq T
\]
\end{theorem}
\begin{theorem}[Bayes' Formula]
\normalfont For $0\leq s\leq t\leq T$, we have
\[
\expec^\mathbb{Q}[X_t\mid \mathcal{F}_s]=\frac{1}{L_s}\expec^\mathbb{P}[L_t\cdot X_t\mid \mathcal{F}_s]
\]
\end{theorem}
%\clearpage
\section{Martingale \& Girsanov}
\begin{definition}[Martingale]
\normalfont Let $(\Omega, \{\mathcal{F}_t\}_{0\leq t\leq T}, \mathbb{P})$ be a filtered probability space. Consider an adapted stochastic process $I_t,0\leq t\leq T$. \\
We say that $I_t$ is a $\mathbb{P}$-\textbf{martingale} if
\[
\expec^\mathbb{P}[I_t\mid \mathcal{F}_s]=I_s\text{  for all }0\leq s\leq t\leq T
\]
\end{definition}
Heuristically, we have
\[
\expec_t[\diff I_t]=0
\]
\begin{theorem}
\normalfont Suppose $X_t$ is an adapted process, and $W_t$ a Brownian motion under $\mathbb{P}$. Define
\[
I_t=\int_0^t X_u\diff W_u
\]
or equivalently,
\[
\diff I_t=X_t\diff W_t
\]
Then $I_t$ is a $\mathbb{P}$-martingale.
\end{theorem}
\begin{theorem}[Martingale Representation Theorem]
\normalfont Suppose $I_t$ is a $\mathbb{P}$-martingale, and $W_t$ a Brownian motion under $\mathbb{P}$. Then there is an adapted process $X_t$ under $\mathbb{P}$, such that
\[
I_t=I_0+\int_0^t X_u\diff W_u
\]
i.e.,
\[
\diff I_t=X_t\diff W_t
\]
\end{theorem}
\begin{theorem}[Girsanov]
\normalfont Suppose $W_t$ is a Brownian motion under measure $\mathbb{P}$, and $\theta$ a constant. Define
\[
\tilde{W}_t=W_t+\theta t
\]
i.e.,
\[
\diff \tilde{W}_t=\diff W_t+\theta \diff t
\]
Then there exists a measure $\mathbb{Q}$, equivalent to $\mathbb{P}$, such that $\tilde{W}_t$ is a $\mathbb{Q}$-Brownian motion.\\
Moreover, the probability $\mathbb{Q}$ is defined by
\[
L_T=\frac{\diff \mathbb{Q}}{\diff \mathbb{P}} = e^{-\frac{1}{2}\theta^2T-\theta W_T}
\]
\end{theorem}
The Radon-Nikodym process $L_t$ in the Girsanov Theorem is given by
\[
L_t=\expec^\mathbb{P}[L_T\mid \mathcal{F}_t]=e^{-\frac{1}{2}\theta^2t-\theta W_t}
\]
By Ito's Lemma, $L_t$ has the following equivalent differential form:
\[
\diff L_t =-\theta L_t\diff W_t
\]
\begin{theorem}[Martinalizing Discounted Stock Price]
\normalfont In $\mathbb{P}$, $\diff S_t = \mu S_t\diff t + \sigma S_t\diff W_t$. We denote $\frac{S_t}{M_t}$ the \textbf{discounted stock price}, where $M_t$ is the money market account. By Ito's Lemma,
\[
\diff(\frac{S_t}{M_t}=\sigma\frac{S_t}{M_t}(\frac{\mu-r}{\sigma}\diff t + \diff W_t)
\]
Therefore, we define
\[
\diff \tilde{W}_t=\frac{\mu-r}{\sigma}\diff t+\diff W_t
\]
to arrive at 
\[
\diff (\frac{S_t}{M_t})=\sigma\frac{S_t}{M_t}\diff \tilde{W}_t
\]
which makes $\frac{S_t}{M_t}$ a $\mathbb{Q}$-martingale for a equivalent probability measure $\mathbb{Q}\sim \mathbb{P}$.\\
Girsanov Theorem ensures $\tilde{W}_t$ is a Brownian motion in $\mathbb{Q}$, given by the Radon-Nikodym process
\[
\diff L_t = -\theta L_t\diff W_t
\]
where $\theta=\frac{\mu-r}{\sigma}$ is the Sharpe ratio.\\
The dynamic of $S_t$ in $\mathbb{Q}$ is
\[
\diff S_t = rS_t\diff t +\sigma S_t\diff \tilde{W}_t
\]
so
\[
S_t=S_0e^{(r-\frac{1}{2}\sigma^2)t+\sigma\tilde{W}_t}
\]
\end{theorem}
\subsection{Risk Neutral Valuation}
We often denote $\mathbb{Q}$ as the risk neutral measure.
\begin{definition}[European Contingent Claim]
\normalfont A \textbf{European contingent claim} or a $T$-claim is a financial instrument consisting of a payment $V_T$ at maturity date $T$. Here $V_T$ is non-negative random variable.
\end{definition}
\begin{theorem}[Risk-Neutral Valuation Formula]
\normalfont \[
\frac{V_t}{M_t}=\expec_t^\mathbb{Q}[\frac{V_T}{M_T}]
\]
\end{theorem}
Below is an outline of derivation:
\begin{enumerate}
  \item Define $U_t:=\expec_t^\mathbb{Q}[\frac{V_T}{M_T}]$. Note $U_t$ is a $\mathbb{Q}$ martingale.
  \item Martingale Representation Theorem suggests there exists some adapted process $\eta_t$ such that
  \[
\diff U_t=\eta_t\diff \tilde{W}_t
  \]
  \item We already have $\diff (\frac{S_t}{M_t})=\sigma\frac{S_t}{M_t}\diff \tilde{W}_t$. Therefore, $\diff U_t=\phi_t\diff(\frac{S_t}{M_t})$ where $\phi_t=\frac{\eta_t M_t}{\sigma S_t}$.
  \item We define $\Pi_t=\phi_tS_t+\gamma_tM_t$ where $\gamma_t=U_t-\phi_t\frac{S_t}{M_t}$. Therefore, $\Pi_t=U_tM_t$ and $\Pi_T=U_TM_T=V_T$. We claim $U_tM_t$ is arbitrage-free price of derivative.
  \item Since $\Pi$ is self financing, we can show $\diff \Pi_t=\diff(U_tM_t)=\diff(U_t e^{rt})$. Applying Ito's Lemma, $\diff \Pi_t = rU_tM_t\diff t + e^{rt}\phi_t\diff (\frac{S_t}{M_t})$. Applying Ito's Lemma one more time and we can arrive at $\Pi_t=\phi_t\diff S_t+r_t\diff M_t$, which is the definition of self-financing condition.
  \item Then it follows $U_tM_t=V_t$, and theorem follows.
\end{enumerate}
Using this theorem, we can calculate the derivative price at time $t$ from its final payoff at time $T$, by taking expectation
\[
V_t=\expec^\mathbb{Q}[e^{-r(T-t)}V_T\mid \mathcal{F}_t]
\]
where one can write $V_T=V_t\cdot f(T-t)$ to get rid of conditional expectation, and take integral eventually after finding out the upper/lower bound of the integral.
\section{Useful Properties}
\begin{theorem}[Lognormal Distribution]
Suppose $X$ follows lognormal distribution$(\mu, \sigma^2)$, then the mean is $\exp(\mu+\sigma^2/2)$ and variance is $(\exp(\sigma^2)-1)\exp(2\mu+\sigma^2)$.
\end{theorem}
\end{document}
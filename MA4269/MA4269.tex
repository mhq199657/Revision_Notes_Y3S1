\PassOptionsToPackage{svgnames}{xcolor}
\documentclass[12pt]{article}



\usepackage[margin=1in]{geometry}  
\usepackage{graphicx}             
\usepackage{amsmath}              
\usepackage{amsfonts}              
\usepackage{framed}               
\usepackage{amssymb} 
\usepackage{array}
\usepackage{amsthm}
\usepackage[nottoc]{tocbibind}
\usepackage{bm}
\usepackage{enumitem}


  \newcommand\norm[1]{\left\lVert#1\right\rVert}
\setlength{\parindent}{0cm}
\setlength{\parskip}{0em}
\newcommand{\Lim}[1]{\raisebox{0.5ex}{\scalebox{0.8}{$\displaystyle \lim_{#1}\;$}}}
\newtheorem{definition}{Definition}[section]
\newtheorem{theorem}{Theorem}[section]
\newtheorem{notation}{Notation}[section]
\theoremstyle{definition}
\DeclareMathOperator{\arcsec}{arcsec}
\DeclareMathOperator{\arccot}{arccot}
\DeclareMathOperator{\arccsc}{arccsc}
\DeclareMathOperator{\PV}{PV}
\DeclareMathOperator{\TV}{TV}
\DeclareMathOperator{\diff}{d}
\DeclareMathOperator{\expec}{E}
\DeclareMathOperator{\var}{Var}
\DeclareMathOperator{\cov}{Cov}
\DeclareMathOperator{\CE}{CE}
\DeclareMathOperator{\RP}{RP}
\newcommand\cf[1]{\mathbf{#1}}
\setcounter{tocdepth}{1}
\setcounter{section}{-1}
\begin{document}

\title{Revision notes - MA4269}
\author{Ma Hongqiang}
\maketitle
\tableofcontents

\clearpage
%\twocolumn
\section{Review}
\subsection{Introduction}
\begin{definition}[Zero-coupon Bond]
\hfill\\\normalfont A contract to deliver $\$1$ on a future date $T$ is known as a \textbf{zero-coupon bond}.\\
The price of the bond at time $t<T$ is denoted by $P(t,T)$, where $T$ is the maturity of the bond.
\end{definition}
\begin{definition}[Money Market Account]
\hfill\\\normalfont The \textbf{money market account} is an asset created by the following procedure, called continuously compounded:
\begin{itemize}
  \item The initial amount equal to $\$1$ is invested at time $t=0$ in the bond with the shortest available maturity(i.e. the next infinitesimal instant).
  \item The position is rolled over to the bond with the next shortest maturity once the first bond expires.
\end{itemize}
The price of the money market at time $t$ is denoted by $M_t$.
\end{definition}
\begin{theorem}
\hfill\\\normalfont We have
\[
\diff M_t = r(t)M_t\diff t
\]
to describe the value of money market account, where $r(t)$ is a parameter known as the interest rate.\\
Solving, we have
\[
M_t = \exp\left(\int_0^t r(u)\diff u\right)
\]
\end{theorem}
\textbf{Remark}: The relationship between $M_t$ and $P(t,T)$ is non-trivial if $r(t)$ is stochastic. However, 
\begin{theorem}
\hfill\\\normalfont Suppose interest rate $r$ is constant. Then, 
\[
P(t,T)=\frac{M_t}{M_T} = e^r(T-t)
\]
It is obvious that $P(t,T)<1$ as long as $r>0$.
\end{theorem}
\begin{definition}[Value of Asset]
\hfill\\\normalfont The \textbf{value} of an asset is the amount of dollars that an investor will pay to own that asset.\\
The term \textbf{stock price} refers to the value (per unit) of the stock under consideration.
\end{definition}
\begin{definition}[Position]
\hfill\\\normalfont A \textbf{position} in an asset is the equantity of an asset owned or owed by an investor.
\begin{itemize}
  \item \textbf{long position}: the investor \textit{owns} the asset.
  \item \textbf{short position}: the investor \textit{sells} the asset that he does not own.
\end{itemize}
\end{definition}
\begin{definition}[Portfolio]
\hfill\\\normalfont A \textbf{portfolio} is a combination of various positions in financial assets.\\
At any time $t$, the \textbf{value of the portfolio} $\Pi_t$ is just eh sum of the values of all the positions held in the portfolio at that particular time $t$:
\[
\Pi_t = a_t^{(1)}A_t^{(1)}+\cdots +a_t^{(n)}A_t^{(n)}
\]
where $a_t^{(i)}$ is the position where $A_t^{(i)}$ is the price of the asset at time $t$.\\
Since the price of assets is not determined by us, therefore, $\Pi_t$ can be written as the vector 
\[
(a_t^{(1)}),\ldots, a_t^{(n)})
\]
\end{definition}
\begin{definition}[Self-financing]
\hfill\\\normalfont A portfolio $\Pi_t=(a_t^{(1)}),\ldots, a_t^{(n)})$ is \textbf{self-financing} over the time interval $[0,T]$ if there is \textit{no} exogeneous infusion or withdrawal of money, \textit{except} possibly at the initiation time $0$ or maturity date $T$:
\[
\diff\Pi_t = a_t^{(1)}dA_t^{(1)}+\cdots +a_t^{(n)}dA_t^{(n)}
\]
Roughly speaking, the differential equation says tha thte change in portfolio is completely due to the change in the underlying asset prices and nothing else.
\end{definition}
\begin{definition}[Direction of Cash Flow]
\hfill\\\normalfont Suppose I am an investor
\begin{itemize}
  \item If cash flow is positive, it measn that someone pays me.
  \item If cash flow is negative, it means that I pay someone.
\end{itemize}
\end{definition}
Cash flow depends on
\begin{enumerate}
  \item the value of the underlying portfolio, and
  \item whether or not the portfolio is entered into or liquidated.
\end{enumerate}
\subsection{Financial Market}
\begin{definition}[Arbitrage]
\hfill\\\normalfont An \textbf{arbitrage opportunity} is the existence of a self-financing portfolio $\Pi_t, 0\leq t\leq T$, having the following properties:
\begin{enumerate}
  \item $\Pi_0=0$
  \item $\Pi_T\geq 0$ for \textbf{all possible} outcomes
  \item There is a positive probability that $\Pi_T>0$.
\end{enumerate}
\end{definition}
\begin{definition}[Equivalent Definition of Arbitrage]
\hfill\\\normalfont An arbitrage opportunity is the existence of a self-financing portfolio $\Pi_t, 0\leq t\leq T$, having the following properties:
\begin{enumerate}
  \item $\Pi_T-\Pi_0e^{rT}\geq 0$ for \textbf{all possible} outcomes
  \item There is a positive probability that $\Pi_T-\Pi_0e^{rT}>0$
\end{enumerate}
\end{definition}
\begin{theorem}[Consequence of No Arbitrage]
\hfill\\\normalfont Suppose there are two self-financing portfolios $\Pi_t^A$ and $\Pi_t^B$ over the time interval $[t,T]$ such that $\Pi_T^A\geq \Pi_T^B$. Then in the absence of arbitrage, we must have
\[
\Pi_t^A\geq \Pi_t^B
\]
\end{theorem}
\begin{theorem}[Law of One Price]
\hfill\\\normalfont Law of one price is a consequence of the above theorem.\\
Suppose there are two self-financing portfolios $\Pi_t^A$ and $\Pi_t^B$ over the time interval $[t,T]$ such that $\Pi_T^A = \Pi_T^B$. Then, in the absence of arbitrage, we must have
\[
\Pi_t^A = \Pi_t^B
\]
\end{theorem}
\begin{theorem}
\hfill\\\normalfont All \textit{risk-free} portfolios must earn the same return, i.e., \textit{riskless interest rate}. Suppose $\Pi_t$ is teh valueo f a riskfree portfolio, and $\diff\Pi_t$ is the price increment during a small period of time interval $[t,t+\diff t]$. Then
\[
\diff \Pi_t = r\Pi_t\diff t
\]
where $r$ is the riskless interest rate.
\end{theorem}
\subsection{Forward Contracts \& Options}
\begin{definition}[Forward Contract]
\hfill\\\normalfont A \textbf{forward contract} is a contract that delivers one unit of the underlying asset on a known future date $T$ for a certain price $K$ agreed today.\\
Here,
\begin{itemize}
  \item $K$ is the \textbf{delivery price}
  \item $T$ is called the \textbf{delivery date}
  \item the buyer of the contract is in the long position
  \item the seller of the contract is in the short position
  \item the delivery price $K$ is the amount the long sides pays the short side in exchange of one unit of the \textbf{udnerlying asset whose value} is $S_T$ on the delivery date $T$.
\end{itemize}
\end{definition}
\begin{definition}[Forward Price]
\hfill\\\normalfont The \textbf{forward price} at time $t$ is the delivery price of a forward contract which costs nothing to enter into at time $t$.
We denote the forward price at time $t$ by
\[
F(S_t, t, T)
\]
\end{definition}
\textbf{Remark}: The forward price $F(S,t,T)$ is \textit{not} the value of corresponding forward contract.
\begin{definition}[Payoff, Profit]
\hfill\\\normalfont The \textbf{payoff} to a position is the value of the position at the maturity date $T$.\\
The \textbf{profit} to a position is the payoff to the position at maturity dat $T$, subtracted by the time-$T$ value of the initial investment in the position:
\[
\Pi_T-\Pi_te^{r(T-t)}
\]
\end{definition}
\begin{theorem}[Payoff of Forward Contract]
\hfill\\\normalfont It is obvious from the definition that 
\begin{itemize}
  \item the payoff to a long forward contract is $S_T-K$;
  \item the payoff to a short forward contract is $K-S_T$
\end{itemize}
\end{theorem}
Suppose it costs nothing to enter into a forward contract, then by using the forward price definition, 
\begin{itemize}
  \item the payoff and the profit to a long forward contract are the same:
  \[
S_T-F(S, t, T)
  \]
  \item the payoff and profit to a short forward contract are the same:
  \[
F(S,t,T) - S_T
  \]
\end{itemize}
\begin{theorem}[Forward Price]
\hfill\\\normalfont Suppose the underlying stock $S$ does not pay dividends. Then the forward price $F(S,t, T)$ of stock at time $t$ is given by
\[
F(S,t,T) = Se^{r(T-t)}
\]
where $S$ is the price of the stock at time $t$.
\end{theorem}
\end{document}
\PassOptionsToPackage{svgnames}{xcolor}
\documentclass[12pt]{article}



\usepackage[margin=0.1in]{geometry}  
\usepackage{graphicx}             
\usepackage{amsmath}              
\usepackage{amsfonts}              
\usepackage{framed}               
\usepackage{amssymb} 
\usepackage{array}
\usepackage{amsthm}
\usepackage[nottoc]{tocbibind}
\usepackage{bm}
\usepackage{enumitem}


  \newcommand\norm[1]{\left\lVert#1\right\rVert}
\setlength{\parindent}{0cm}
\setlength{\parskip}{0em}
\newcommand{\Lim}[1]{\raisebox{0.5ex}{\scalebox{0.8}{$\displaystyle \lim_{#1}\;$}}}
\newtheorem{definition}{Definition}[section]
\newtheorem{theorem}{Theorem}[section]
\newtheorem{notation}{Notation}[section]
\theoremstyle{definition}
\DeclareMathOperator{\arcsec}{arcsec}
\DeclareMathOperator{\arccot}{arccot}
\DeclareMathOperator{\arccsc}{arccsc}
\DeclareMathOperator{\PV}{PV}
\DeclareMathOperator{\TV}{TV}
\DeclareMathOperator{\diff}{d}
\DeclareMathOperator{\expec}{E}
\DeclareMathOperator{\var}{Var}
\DeclareMathOperator{\cov}{Cov}
\DeclareMathOperator{\CE}{CE}
\DeclareMathOperator{\RP}{RP}
\newcommand\cf[1]{\mathbf{#1}}
\setcounter{tocdepth}{1}
\setcounter{section}{-1}
\begin{document}

\title{Revision notes - MA4269}
\author{Ma Hongqiang}
\maketitle
\tableofcontents


\twocolumn
\section{Review}
\subsection{Introduction}
\begin{definition}[Zero-coupon Bond]
\hfill\\\normalfont A contract to deliver $\$1$ on a future date $T$ is known as a \textbf{zero-coupon bond}.\\
The price of the bond at time $t<T$ is denoted by $P(t,T)$, where $T$ is the maturity of the bond.
\end{definition}
\begin{definition}[Money Market Account]
\hfill\\\normalfont The \textbf{money market account} is an asset created by the following procedure, called continuously compounded:
\begin{itemize}
  \item The initial amount equal to $\$1$ is invested at time $t=0$ in the bond with the shortest available maturity(i.e. the next infinitesimal instant).
  \item The position is rolled over to the bond with the next shortest maturity once the first bond expires.
\end{itemize}
The price of the money market at time $t$ is denoted by $M_t$.
\end{definition}
\begin{theorem}
\hfill\\\normalfont We have
\[
\diff M_t = r(t)M_t\diff t
\]
to describe the value of money market account, where $r(t)$ is a parameter known as the interest rate.\\
Solving, we have
\[
M_t = \exp\left(\int_0^t r(u)\diff u\right)
\]
\end{theorem}
\textbf{Remark}: The relationship between $M_t$ and $P(t,T)$ is non-trivial if $r(t)$ is stochastic. However, 
\begin{theorem}
\hfill\\\normalfont Suppose interest rate $r$ is constant. Then, 
\[
P(t,T)=\frac{M_t}{M_T} = e^r(T-t)
\]
It is obvious that $P(t,T)<1$ as long as $r>0$.
\end{theorem}
\begin{definition}[Value of Asset]
\hfill\\\normalfont The \textbf{value} of an asset is the amount of dollars that an investor will pay to own that asset.\\
The term \textbf{stock price} refers to the value (per unit) of the stock under consideration.
\end{definition}
\begin{definition}[Position]
\hfill\\\normalfont A \textbf{position} in an asset is the equantity of an asset owned or owed by an investor.
\begin{itemize}
  \item \textbf{long position}: the investor \textit{owns} the asset.
  \item \textbf{short position}: the investor \textit{sells} the asset that he does not own.
\end{itemize}
\end{definition}
\begin{definition}[Portfolio]
\hfill\\\normalfont A \textbf{portfolio} is a combination of various positions in financial assets.\\
At any time $t$, the \textbf{value of the portfolio} $\Pi_t$ is just eh sum of the values of all the positions held in the portfolio at that particular time $t$:
\[
\Pi_t = a_t^{(1)}A_t^{(1)}+\cdots +a_t^{(n)}A_t^{(n)}
\]
where $a_t^{(i)}$ is the position where $A_t^{(i)}$ is the price of the asset at time $t$.\\
Since the price of assets is not determined by us, therefore, $\Pi_t$ can be written as the vector 
\[
(a_t^{(1)}),\ldots, a_t^{(n)})
\]
\end{definition}
\begin{definition}[Self-financing]
\hfill\\\normalfont A portfolio $\Pi_t=(a_t^{(1)}),\ldots, a_t^{(n)})$ is \textbf{self-financing} over the time interval $[0,T]$ if there is \textit{no} exogeneous infusion or withdrawal of money, \textit{except} possibly at the initiation time $0$ or maturity date $T$:
\[
\diff\Pi_t = a_t^{(1)}dA_t^{(1)}+\cdots +a_t^{(n)}dA_t^{(n)}
\]
Roughly speaking, the differential equation says tha thte change in portfolio is completely due to the change in the underlying asset prices and nothing else.
\end{definition}
\begin{definition}[Direction of Cash Flow]
\hfill\\\normalfont Suppose I am an investor
\begin{itemize}
  \item If cash flow is positive, it measn that someone pays me.
  \item If cash flow is negative, it means that I pay someone.
\end{itemize}
\end{definition}
Cash flow depends on
\begin{enumerate}
  \item the value of the underlying portfolio, and
  \item whether or not the portfolio is entered into or liquidated.
\end{enumerate}
\subsection{Financial Market}
\begin{definition}[Arbitrage]
\hfill\\\normalfont An \textbf{arbitrage opportunity} is the existence of a self-financing portfolio $\Pi_t, 0\leq t\leq T$, having the following properties:
\begin{enumerate}
  \item $\Pi_0=0$
  \item $\Pi_T\geq 0$ for \textbf{all possible} outcomes
  \item There is a positive probability that $\Pi_T>0$.
\end{enumerate}
\end{definition}
\begin{definition}[Equivalent Definition of Arbitrage]
\hfill\\\normalfont An arbitrage opportunity is the existence of a self-financing portfolio $\Pi_t, 0\leq t\leq T$, having the following properties:
\begin{enumerate}
  \item $\Pi_T-\Pi_0e^{rT}\geq 0$ for \textbf{all possible} outcomes
  \item There is a positive probability that $\Pi_T-\Pi_0e^{rT}>0$
\end{enumerate}
\end{definition}
\begin{theorem}[Consequence of No Arbitrage]
\hfill\\\normalfont Suppose there are two self-financing portfolios $\Pi_t^A$ and $\Pi_t^B$ over the time interval $[t,T]$ such that $\Pi_T^A\geq \Pi_T^B$. Then in the absence of arbitrage, we must have
\[
\Pi_t^A\geq \Pi_t^B
\]
\end{theorem}
\begin{theorem}[Law of One Price]
\hfill\\\normalfont Law of one price is a consequence of the above theorem.\\
Suppose there are two self-financing portfolios $\Pi_t^A$ and $\Pi_t^B$ over the time interval $[t,T]$ such that $\Pi_T^A = \Pi_T^B$. Then, in the absence of arbitrage, we must have
\[
\Pi_t^A = \Pi_t^B
\]
\end{theorem}
\begin{theorem}
\hfill\\\normalfont All \textit{risk-free} portfolios must earn the same return, i.e., \textit{riskless interest rate}. Suppose $\Pi_t$ is teh valueo f a riskfree portfolio, and $\diff\Pi_t$ is the price increment during a small period of time interval $[t,t+\diff t]$. Then
\[
\diff \Pi_t = r\Pi_t\diff t
\]
where $r$ is the riskless interest rate.
\end{theorem}
\subsection{Forward Contracts \& Options}
\begin{definition}[Forward Contract]
\hfill\\\normalfont A \textbf{forward contract} is a contract that delivers one unit of the underlying asset on a known future date $T$ for a certain price $K$ agreed today.\\
Here,
\begin{itemize}
  \item $K$ is the \textbf{delivery price}
  \item $T$ is called the \textbf{delivery date}
  \item the buyer of the contract is in the long position
  \item the seller of the contract is in the short position
  \item the delivery price $K$ is the amount the long sides pays the short side in exchange of one unit of the \textbf{udnerlying asset whose value} is $S_T$ on the delivery date $T$.
\end{itemize}
\end{definition}
\begin{definition}[Forward Price]
\hfill\\\normalfont The \textbf{forward price} at time $t$ is the delivery price of a forward contract which costs nothing to enter into at time $t$.
We denote the forward price at time $t$ by
\[
F(S_t, t, T)
\]
\end{definition}
\textbf{Remark}: The forward price $F(S,t,T)$ is \textit{not} the value of corresponding forward contract.
\begin{definition}[Payoff, Profit]
\hfill\\\normalfont The \textbf{payoff} to a position is the value of the position at the maturity date $T$.\\
The \textbf{profit} to a position is the payoff to the position at maturity dat $T$, subtracted by the time-$T$ value of the initial investment in the position:
\[
\Pi_T-\Pi_te^{r(T-t)}
\]
\end{definition}
\begin{theorem}[Payoff of Forward Contract]
\hfill\\\normalfont It is obvious from the definition that 
\begin{itemize}
  \item the payoff to a long forward contract is $S_T-K$;
  \item the payoff to a short forward contract is $K-S_T$
\end{itemize}
\end{theorem}
Suppose it costs nothing to enter into a forward contract, then by using the forward price definition, 
\begin{itemize}
  \item the payoff and the profit to a long forward contract are the same:
  \[
S_T-F(S, t, T)
  \]
  \item the payoff and profit to a short forward contract are the same:
  \[
F(S,t,T) - S_T
  \]
\end{itemize}
\begin{theorem}[Forward Price]
\hfill\\\normalfont Suppose the underlying stock $S$ does not pay dividends. Then the forward price $F(S,t, T)$ of stock at time $t$ is given by
\[
F(S,t,T) = Se^{r(T-t)}
\]
where $S$ is the price of the stock at time $t$.
\end{theorem}
\begin{definition}[Call Option]
\hfill\\\normalfont A \textbf{call option} is an agreement where the buyer has the \textit{right}, but not the obligation to buy the underlying asset, for a certain price $K$ agreed at the initiation of the contract. Here, $K$ is called the stike price, whereby $T$ is used to denote maturity, which is the darte by which option must be exercised or it becomes worthless. \\
For now, we consider only European call option, where exercise of the contract occurs only at maturity $T$.\\
The payoff to a long European call option with strike price $K$ and maturity $T$ is
\[
(S_T-K)^+=\max\{S_T-K,0\}
\]
where the payoff to a short European call option with same strike price and maturity is $-(S_T-K)^+$.
\end{definition}
\begin{definition}[Put Option]
\hfill\\\normalfont A \textbf{put option} is an agreement where the buyer has the right to sell an asset, but not the obligation to sell, for a certain price $K$ agreed at the initiation of the contract.\\
The payoff to a long European put option with strike price $K$ and expiration $T$ is
\[
(K-S_T)^+=\max\{K-S_T,0\}
\]
whereas the payoff to a short European put option with same strike price and expiration $T$ is $-(K-S_T)^+$.
\end{definition}
\begin{definition}[Moneyness]
\hfill\\\normalfont Options are often described by their degree of moneyness. At any time $t$, an option is said to be 
\begin{itemize}
  \item \textbf{in-the-money} if payoff at time $t>0$.
  \item \textbf{at-the-money} if payoff$=0$, i.e., $S_t=K$.
  \item \textbf{out-of-the-money} if payoff$<0$.
\end{itemize}
\end{definition}
\begin{theorem}[Put Call Parity]
\hfill\\\normalfont We have the following relationship between call $c$ and put $p$ price, over the underlying asset at time $t$. Here $K$ is the strike price of the options and $F$ is the forward price at time $t$.
\[
c-p+(K-F)e^{-r(T-t)}=0
\]
\end{theorem}
\subsection{Binomial Model}
\begin{definition}[One-period Binomial Model]
\hfill\\\normalfont Suppose the non-dividend paying stock price per share today is $S_0$. We assume that, at the end of the one period, the stock price is either $S_0u$ or $S_0d$ where $d$ and $u$ are positive real numbers such that $d<u$.\\
We call $u$ the \textbf{up factor} and $d$ the \textbf{down factor}.\\
Consider a derivative on the stock with time $T$ to maturity. Let $V_0$ be the price of derivative at time $0$.\\
We can price $V_0$ by constructing $\Pi_0=V_0-\phi S_0$, and make it riskless, i.e. $\Pi_T = V_u-\phi S_0u=V_d-\phi S_0d$, by picking a suitable $\phi$. Since $\Pi_0$ is riskless, its payoff should be the same as any other riskless payoff, e.g. money market account,i.e., $\Pi_T = (V_0-\phi S_0)e^{rT}$.\\
Solving, we have
\[
V_0 = e^{-rT}(pV_u+(1-p)V_d)
\]
where $p=\frac{e^{rT}-d}{u-d}$.
\end{definition}
We can interpret $p$ and $1-p$ as probabilities distribution on $S_T$, so that we can write
\[
V_0 = e^{-rT}E^\mathbb{Q}[V_T]
\]
The expectation of $S_T$ under $\mathbb{Q}$ is
\[
E^\mathbb{Q}[S_T] = S_0e^{rT}
\]
which matches our riskless argument.
\begin{theorem}[Restriction on {$u$} and {$d$}]
\hfill\\\normalfont In the one-period binomial model where the one period is $[0,T]$ and the corresponding up-factor and down-factor of a non-dividend paying stock are $u$ and $d$ respectively with $d<u$, we have
\[
d<e^{rT}<u
\]
\end{theorem}
\begin{definition}[Multi-period Binomial Model]
\hfill\\\normalfont At any time $j\Delta t$, there are $j+1$ possible stock prices:
\[
S_0d^j, S_0d^{j-1}u,\ldots, S_0u^j
\]
Without loss of generality, we can assume that $ud=1$.\\
If $V_j^k$ is the price of the derivative at time $j\Delta t$ when the underlying stock price is $S_0d^{j-k}u^k$, i.e. there are $k$ period out of $j$ that the price goes up.\\
We then have
\[
V_0 = e^{-rn\Delta t}\sum_{i=0}^n\binom{n}{i}p^i(1-p)^{n-i}V_n^i
\]
where $V_0$ is the price of the European style derivative given by the $n$ period binomial model.
\end{definition}

\section{Brownian Motion}
\subsection{Brownian Motion}
\begin{definition}[Standard Brownian Motion]
\hfill\\\normalfont A \textbf{standard Brownian Motion} is a stochastic process $W_t, t\geq 0$ with the following defining characteristics:
\begin{itemize}
  \item[(W1)] $W_0=0$.
  \item[(W2)] With probability $1$ (almost surely), the function $t\to W_t$ is continuous in $t$.
  \item[(W3)] For every $0\leq t_1<t_2$, $W_{t_2}-W_{t_1}$ is normally distributed with mean $0$ and variance $t_2-t_1$.
  \item[(W4)] $W_{t_3}-W_{t_2}$ is independent of $W_{t_1}-W_{t_0}$ for any $0\leq t_0\leq t_1\leq t_2\leq t_3$, i.e. non-overlapping increments are independently distributed.
\end{itemize}
\end{definition}
Property (W3) implies that for any $\Delta t$, $W_{t+\Delta t}-W_t\sim N(0,\Delta t)$. This implies that $|W_t|\leq 1.96\sqrt{t}$ with 95\% probability.\\
Property (W4) implies that $\cov (W_t, W_s) = E[W_tW_s] = \min\{t,s\}$.\footnote{We write $W_t=(W_t-W_s+W_s)$ if $t>s$.}
\begin{theorem}[Binomial Approximation to Brownian Motion]
\hfill\\\normalfont Let $\varepsilon_1,\ldots$ be a sequence of independent, identically distributed random variables with mean $0$ and variance $1$. For each $n\geq 1$, define a continuous time stochastic process $W_t^{(n)}$ by
\[
W_t^{(n)} = \frac{1}{\sqrt{n}} \sum_{1\leq i\leq [nt]} \varepsilon_i
\]
$W_t^{(n)}$ approaches a standard Brownian motion $N(0,t)$ as $n\to \infty$.
\end{theorem}
\begin{theorem}[Infinitesimal Brownian Increment]
\hfill\\\normalfont We can approximate $\diff W_t$ is a very small time interval $\Delta t$:
\[
\Delta W_t:=W_{t+\Delta t} - W_t
\]
and then we have
\[
\Delta W_t = \phi \sqrt{\Delta t} \text{i.e., }\Delta W_t \sim N(0, \Delta t)
\]
where $\phi$ has a standard normal distribution.
\end{theorem}
\begin{definition}[Generalised Wiener Process]
\hfill\\\normalfont A \textbf{generalised Wiener Process} for a variable $X$ can be defined in terms of $\diff W_t$ as
\[
\diff X_t = a\diff t + b\diff W_t
\]
where $a$ and $b$ are constants. The parameter $a$ is called the \textbf{drift rate} and $b^2$ is called the \textbf{variance rate} of the process.\\
In a small time interval $\Delta t$, the change $\Delta X_t$ is given by
\[
\Delta X_t = a\Delta t + b\Delta W_t
\]
Therefore, $\Delta X_t$ has a normal distribution with mean $a\Delta t$ and variance $b^2\Delta t$.
\end{definition}
Here, we can safely write $X_t = X_0+at+bW_t$.
\subsection{Quadratic Variation}
\begin{definition}[Quadratic Variation]
Any sequence of values $0=t_0<t_1<\cdots<t_n=T$ is called a partition $\Pi=\Pi(t_0,\ldots, t_n)$ of a fixed interval $[0,T]$. The discrete quadratic variation of a standard Brownian motion $W$ relative to the partition $\Pi$ is defined as
\[
Q(W,\Pi) = \sum_{i=1}^n (W_{t_i}-W_{t_{i-1}})^2
\]
For any partition $\Pi$, define
\[
\|\Pi\| = \max_{1\leq i\leq n}|t_i-t_{i-1}|
\]
\end{definition}
\begin{theorem}[{$n$}th Moment of Standard Normal {$Z$}]
\hfill\\\normalfont The $n$th moment of a random variable $X$ is defined to be $E[X^n]$. If $\phi\sim N(0,1)$, then
\[
E[\phi^n] = \begin{cases}
0&\text{ if }n\text{ is odd}\\
\frac{(2k)!}{2^kk!}&\text{ if }n=2k
\end{cases}
\]
In particular, we have $E(\phi^2) = 1$ and $E(\phi^4) = 3$.
\end{theorem}
\begin{theorem}
\hfill\\\normalfont Consider an arbitrary sequence of partitions $\Pi_n$, where $n = 1,2,\ldots$. Suppose $\lim_{n\to \infty} \|\Pi_n\|=0$, then
\[
\lim_{n\to \infty} E(Q(W,\Pi_n)-T)^2] = 0
\]
That is the standard Brownian motion has quadratic variation which is equal to $T$, in the mean square limit.
\end{theorem}
\begin{definition}
\hfill\\\normalfont Define the integral $\int_0^T (\diff W)^2$ by
\[
\lim_{n\to infty} E[\sum_{i=1}^n (W_{t_i}-W_{t_{i-1}})^2 - \int_0^T(\diff W_t)^2]^2 = 0
\]
\end{definition}
However, from quadrativ variation theorem, we have
\[
\lim_{n\to infty} E[\sum_{i=1}^n (W_{t_i}-W_{t_{i-1}})^2 - \int_0^T\diff t]^2 = 0
\]
Therefore,
\[
\int_0^T(\diff W_t)^2=\int_0^T\diff t
\]
In fact, we can write
\[
(\diff W_t)^2 = \diff t
\]
which gives
\[
(\Delta W)^2 \approx \Delta t
\]
in discrete time approximation.
\subsection{It\^{o}'s lemma}
\begin{definition}[It\^{o}'s Process]
\hfill\\\normalfont The It\^{o}'s Process $\diff X_t$ is defined as 
\[
\diff X_t = a(X_t,t)\diff t + b(X_t,t)\diff W_t
\]
\end{definition}
\begin{theorem}[It\^{o}'s Lemma]
\hfill\\\normalfont Let $V(X_t,t)$ be a smooth function of $t$ and of the It\^{o}'s process $X_t$:
\[
\diff X_t = a\diff t + b\diff W_t
\]
for some $a=a(X_t,t)$ and $b=b(X_t,t)$. Then we have
\[
\diff V(X_t,t) = (a\frac{\partial V}{\partial X}+\frac{\partial V}{\partial t}+\frac{1}{2}b^2\frac{\partial^2 V}{\partial X^2})\diff t+b\frac{\partial V}{\partial X}\diff W_t
\]
\end{theorem}
Another version of the Ito's Lemma where we do not have explicit form of $\diff X$ is
\[
\diff V_t = \diff V(X_t,t) = \frac{\partial V}{\partial t}\diff t+\frac{\partial V}{\partial X}\diff X_t + \frac{1}{2}\frac{\partial^2 V}{\partial X^2}(\diff X_t)^2 
\]
\begin{definition}[Ito Integral]
\hfill\\\normalfont In order to have differential form of the SDE $\diff X_t = a(X_t,t)\diff t+b(X_t,t)\diff W_t$, we require $a(X_t,t)$ and $b(X_t,t)$ to be non-anticipative, which means that its value at $t$ can only be available at time $t$.\\
With the above assumption, we can define
\[
\int_0^T a(X_t,t)\diff t = \lim_{n\to \infty}\sum_{i=1}^n a(X_{t_{i-1}}, t_{i-1})(t_i-t_{i-1})
\]
and Ito integral
\[
\int_0^T b(X_t,t)\diff W_t
\]
is the mean-square limit of the sum $\sum_{i=1}^n b(X_{t_{i-1}}, t_{i-1})(W_{t_i}-W_{t_{i-1}})$.
\end{definition}

\section{Black-Scholes Model}
In Black-Scholes Model, we assume the following 2 conditions:
\begin{enumerate}
  \item The money market(riskless asset) $M_t$ is given by
  \[
\diff M_t = rM_t\diff t
  \]
  \item The stock price follows the Geometric Brownian motion:
  \[
\diff S_t = \mu S_t\diff t +\sigma S_t\diff W_t
  \]
  where $\mu$ and $\sigma$ are constants.
\end{enumerate}
We can derive Black Scholes PDE from either delta hedging, where we take $\Pi_t=V_t-\phi_tS_t, \;0\leq t\leq T$, such that $\phi_t$ is chosen to make $\Pi_t$ self-financing and riskless.\\
Self financing condition gives
\[
\diff \Pi_t=\diff V_t - \phi_t\diff S_t
\]
Also, since $\phi_t$ is chosen so that $\Pi_t$ is riskless, we also need
\[
\diff \Pi_t = r\Pi_t\diff t
\]
Here, $\diff V_t$ can be calculated via Ito's lemma and $\diff S_t$ is given in assumption. Solving, we will have
\[
\phi_t = \frac{\partial V}{\partial S}
\]
and
\[
\frac{\partial V}{\partial t}+r\frac{\partial V}{\partial S}S_t +\frac{1}{2}\sigma^2S_t^2\frac{\partial^2 V}{\partial S^2}=rV
\]
Here, we call $\frac{\partial V}{\partial S}$ \textbf{delta} of the derivative.\\
We can derive the Black Scholes PDE by \textit{replication} also, by considering an asset $\Pi_t=a_tS_t+b_tM_t$, which satisfies $\Pi_t=V_t$ for all $t\leq T$. The equality throughout $T$ gives $\diff \Pi_t=\diff V_t$.\\
Similarly, self financing condition gives
\[
\diff \Pi_t=a_t\diff S_t+v_t\diff M_t
\]
where $\diff S_t$ and $\diff M_t$ are readily available.\\
Also, we can conpute $\diff V_t$ via Ito's Lemma:
\[
\diff V_t = (\frac{\partial V}{\partial S}\mu S_t+\frac{\partial V}{\partial t}+\frac{1}{2}\frac{\partial^2V}{\partial S^2}\sigma^2S_t^2)\diff t+\frac{\partial V}{\partial S}\sigma S_t\diff W_t
\]
Then we compare coefficients of $\diff W_t$, arriving at
\[
a_t=\frac{\partial V}{\partial S}
\]
and therefore $b_t$ can be wriiten as
\[
b_t=\frac{1}{M_t}(V_t-\frac{\partial V}{\partial S}S_t)
\]
whereas comparing $\diff t$ and make necessary computation, we can arrive at
\[
\frac{\partial V}{\partial t}+r\frac{\partial V}{\partial S}S_t +\frac{1}{2}\sigma^2S_t^2\frac{\partial^2 V}{\partial S^2}-rV=0
\]
\textbf{Remark}:
\begin{enumerate}
  \item The drift parameter $\mu$ of the stock never enters into the PDE.
  \item To uniquely determine the solution, we must prescribe
  \begin{itemize}
    \item Boundary conditions
    \item Initial, or final conditions
  \end{itemize}
\end{enumerate}
\begin{theorem}[Solution to Black Scholes PDE]
\hfill\\\normalfont For a European call option with a call price $c(S_t,t)$, we can make the following observation:
\begin{enumerate}
  \item Final condition: $c(S_T,T)=\max\{S_T-K, 0\}$
  \item Boundary condition 1: $S_0=0\Rightarrow c(0,t)=0\text{ for all }0\leq t\leq T$.
  \item Boundary condition 2: With $S_t\gg K$, we have $c(S_t,t)\approx S_t$ for all $0\leq t\leq T$.
\end{enumerate}
With these observation, we have, for European call
\[
c_t= S_tN(d_{+})-Ke^{-r\tau}N(d_{-})
\]
For European put:
\[
p_t=Ke^{-r\tau}N(-d_{-})-S_tN(-d_+)
\]
where 
\[
d_{\pm}=\frac{\ln(S_t/K)+(r\pm \sigma^2/2)\tau}{\sigma\sqrt{\tau}}
\]
and
\[
\tau = T-t
\]
\end{theorem}
\begin{theorem}[Black Scholes PDE with Presence of Dividends]
\hfill\\\normalfont With presence of dividends,
\[
\diff \Pi_t = a_t\diff S_t+b_t\diff M_t+a_tqS_t\diff t
\] 
The black scholes PDE becomes
\[
\frac{\partial V}{\partial t}+(r-q)\frac{\partial V}{\partial S}S_t +\frac{1}{2}\sigma^2S_t^2\frac{\partial^2 V}{\partial S^2}-rV=0
\]
\end{theorem}
\subsection{Preliminaries on Martingale Pricing}
\begin{definition}[Equivalent Probability Measure]
\hfill\\\normalfont Suppose there are two probability measures $\mathbb{P}$ and $\mathbb{Q}$ on space $(\Omega, \mathcal{F})$. We say that $\mathbb{P}$ and $\mathbb{Q}$ are \textbf{equivalent}, denoted by $\mathbb{P}\sim \mathbb{Q}$ if
\[
\mathbb{P}(A)>0\Leftrightarrow \mathbb{Q}(A)>0\text{  for all }A\in\mathcal{F}
\]
\end{definition}
Essentially, two equivalent measures agree on \textbf{all} certain and impossible events.\\
We hope to derive an equivalent probability measure $\mathbb{Q}$ from an existing one $\mathbb{P}$. To do this, we require a \textbf{positive} random variable $L$ with property $\expec^{\mathbb{P}}[L]=1$. These two conditions are two defining characteristic of a Radon-Nikodym Derivative $L$.\\
Define $\mathbb{Q}$ by
\[
\mathbb{Q}(A)=\expec^{\mathbb{P}}[L\cdot \mathbb{1}_A]
\]
where $\mathbb{1}_A$ is the indicator random variable for the event $A$.
\begin{theorem}[Radon Nikodym]
\hfill\\\normalfont Consider two probability measures $\mathbb{P}$ and $\mathbb{Q}$ on $(\Omega, \mathcal{F})$. The following are equivalent:
\begin{enumerate}
  \item $\mathbb{P}\sim\mathbb{Q}$
  \item There eixsts a positive random variable $L$ such that for every event $A\in\mathcal{F}$
  \begin{itemize}
    \item $\mathbb{Q}(A)=\expec^\mathbb{P}[L\cdot \mathbf{1}_A]$
    \item $\mathbb{P}(A)=\expec^\mathbb{Q}[\frac{1}{L}\cdot \mathbf{1}_A]$.
  \end{itemize}
\end{enumerate}
Here $L=\frac{\diff \mathbb{Q}}{\diff \mathbb{P}}$, as derived from theorem, is called the Radon Nikodym derivative of $\mathbb{Q}$ wrt $\mathbb{P}$.
\end{theorem}
\begin{theorem}
\hfill\\\normalfont Let $X$ be a random variable. With above notations, we have
\[
\expec^\mathbb{Q}[X]=\expec^\mathbb{P}[L\cdot X]
\]
and
\[
\expec^\mathbb{P}[X]=\expec^\mathbb{Q}[\frac{1}{L}\cdot X]
\]
\end{theorem}
\begin{definition}[Filtration]
\hfill\\\normalfont Let $\{\mathcal{F}_t\}, 0\leq t\leq T$ be a filtration, where $\mathcal{F}_t$ is information available to us at time $t$.\\
Trivially, we have $\mathcal{F}_s\subseteq \mathcal{F}_t$ for $s\leq t$.
\end{definition}
Suppose we now want to move from $(\Omega, \{\mathcal{F}_t\}, \mathbb{P})$ to $(\Omega, \{\mathcal{F}_t\}, \mathbb{Q})$, we require a random variable $L_T$ satisfying the following:
\begin{itemize}
  \item $L_T$ is $\mathcal{F}_T$ measurable, which means that $L_T$ will be known at time $T$.
  \item $L_T$ is positive.
  \item $\expec^\mathbb{P}[L_T]=1$.
\end{itemize}
Define a stochastic process $L_t$ as follows:
\[
L_t=\expec^\mathbb{P}[L_T\mid \mathcal{F}_t], 0\leq t\leq T
\]
The above process is called \textbf{Radon-Nikodym} derivative/likelihood process.
\begin{theorem}
\hfill\\\normalfont Suppose $X_t,0\leq t\leq T$, is an adapted process on $\Omega, \mathcal{F}_t$. We have
\[
\expec^\mathbb{Q}[X_t]=\expec^\mathbb{P}[L_t\cdot X_t], 0\leq t\leq T
\]
\end{theorem}
\begin{theorem}[Bayes' Formula]
\hfill\\\normalfont For $0\leq s\leq t\leq T$, we have
\[
\expec^\mathbb{Q}[X_t\mid \mathcal{F}_s]=\frac{1}{L_s}\expec^\mathbb{P}[L_t\cdot X_t\mid \mathcal{F}_s]
\]
\end{theorem}

\section{Martingale \& Girsanov}
\begin{definition}[Martingale]
\hfill\\\normalfont Let $(\Omega, \{\mathcal{F}_t\}_{0\leq t\leq T}, \mathbb{P})$ be a filtered probability space. Consider an adapted stochastic process $I_t,0\leq t\leq T$. \\
We say that $I_t$ is a $\mathbb{P}$-\textbf{martingale} if
\[
\expec^\mathbb{P}[I_t\mid \mathcal{F}_s]=I_s\text{  for all }0\leq s\leq t\leq T
\]
\end{definition}
Heuristically, we have
\[
\expec_t[\diff I_t]=0
\]
\begin{theorem}
\hfill\\\normalfont Suppose $X_t$ is an adapted process, and $W_t$ a Brownian motion under $\mathbb{P}$. Define
\[
I_t=\int_0^t X_u\diff W_u
\]
or equivalently,
\[
\diff I_t=X_t\diff W_t
\]
Then $I_t$ is a $\mathbb{P}$-martingale.
\end{theorem}
\begin{theorem}[Martingale Representation Theorem]
\hfill\\\normalfont Suppose $I_t$ is a $\mathbb{P}$-martingale, and $W_t$ a Brownian motion under $\mathbb{P}$. Then there is an adapted process $X_t$ under $\mathbb{P}$, such that
\[
I_t=I_0+\int_0^t X_u\diff W_u
\]
i.e.,
\[
\diff I_t=X_t\diff W_t
\]
\end{theorem}
\begin{theorem}[Girsanov]
\hfill\\\normalfont Suppose $W_t$ is a Brownian motion under measure $\mathbb{P}$, and $\theta$ a constant. Define
\[
\tilde{W}_t=W_t+\theta t
\]
i.e.,
\[
\diff \tilde{W}_t=\diff W_t+\theta \diff t
\]
Then there exists a measure $\mathbb{Q}$, equivalent to $\mathbb{P}$, such that $\tilde{W}_t$ is a $\mathbb{Q}$-Brownian motion.\\
Moreover, the probability $\mathbb{Q}$ is defined by
\[
L_T=\frac{\diff \mathbb{Q}}{\diff \mathbb{P}} = e^{-\frac{1}{2}\theta^2T-\theta W_T}
\]
\end{theorem}
The Radon-Nikodym process $L_t$ in the Girsanov Theorem is given by
\[
L_t=\expec^\mathbb{P}[L_T\mid \mathcal{F}_t]=e^{-\frac{1}{2}\theta^2t-\theta W_t}
\]
By Ito's Lemma, $L_t$ has the following equivalent differential form:
\[
\diff L_t =-\theta L_t\diff W_t
\]
\begin{theorem}[Martinalizing Discounted Stock Price]
\hfill\\\normalfont In $\mathbb{P}$, $\diff S_t = \mu S_t\diff t + \sigma S_t\diff W_t$. We denote $\frac{S_t}{M_t}$ the \textbf{discounted stock price}, where $M_t$ is the money market account. By Ito's Lemma,
\[
\diff(\frac{S_t}{M_t}=\sigma\frac{S_t}{M_t}(\frac{\mu-r}{\sigma}\diff t + \diff W_t)
\]
Therefore, we define
\[
\diff \tilde{W}_t=\frac{\mu-r}{\sigma}\diff t+\diff W_t
\]
to arrive at 
\[
\diff (\frac{S_t}{M_t})=\sigma\frac{S_t}{M_t}\diff \tilde{W}_t
\]
which makes $\frac{S_t}{M_t}$ a $\mathbb{Q}$-martingale for a equivalent probability measure $\mathbb{Q}\sim \mathbb{P}$.\\
Girsanov Theorem ensures $\tilde{W}_t$ is a Brownian motion in $\mathbb{Q}$, given by the Radon-Nikodym process
\[
\diff L_t = -\theta L_t\diff W_t
\]
where $\theta=\frac{\mu-r}{\sigma}$ is the Sharpe ratio.\\
The dynamic of $S_t$ in $\mathbb{Q}$ is
\[
\diff S_t = rS_t\diff t +\sigma S_t\diff \tilde{W}_t
\]
so
\[
S_t=S_0e^{(r-\frac{1}{2}\sigma^2)t+\sigma\tilde{W}_t}
\]
\end{theorem}
\subsection{Risk Neutral Valuation}
We often denote $\mathbb{Q}$ as the risk neutral measure.
\begin{definition}[European Contingent Claim]
\hfill\\\normalfont A \textbf{European contingent claim} or a $T$-claim is a financial instrument consisting of a payment $V_T$ at maturity date $T$. Here $V_T$ is non-negative random variable.
\end{definition}
\begin{theorem}[Risk-Neutral Valuation Formula]
\hfill\\\normalfont \[
\frac{V_t}{M_t}=\expec_t^\mathbb{Q}[\frac{V_T}{M_T}]
\]
\end{theorem}
Below is an outline of derivation:
\begin{enumerate}
  \item Define $U_t:=\expec_t^\mathbb{Q}[\frac{V_T}{M_T}]$. Note $U_t$ is a $\mathbb{Q}$ martingale.
  \item Martingale Representation Theorem suggests there exists some adapted process $\eta_t$ such that
  \[
\diff U_t=\eta_t\diff \tilde{W}_t
  \]
  \item We already have $\diff (\frac{S_t}{M_t})=\sigma\frac{S_t}{M_t}\diff \tilde{W}_t$. Therefore, $\diff U_t=\phi_t\diff(\frac{S_t}{M_t})$ where $\phi_t=\frac{\eta_t M_t}{\sigma S_t}$.
  \item We define $\Pi_t=\phi_tS_t+\gamma_tM_t$ where $\gamma_t=U_t-\phi_t\frac{S_t}{M_t}$. Therefore, $\Pi_t=U_tM_t$ and $\Pi_T=U_TM_T=V_T$. We claim $U_tM_t$ is arbitrage-free price of derivative.
  \item Since $\Pi$ is self financing, we can show $\diff \Pi_t=\diff(U_tM_t)=\diff(U_t e^{rt})$. Applying Ito's Lemma, $\diff \Pi_t = rU_tM_t\diff t + e^{rt}\phi_t\diff (\frac{S_t}{M_t})$. Applying Ito's Lemma one more time and we can arrive at $\Pi_t=\phi_t\diff S_t+r_t\diff M_t$, which is the definition of self-financing condition.
  \item Then it follows $U_tM_t=V_t$, and theorem follows.
\end{enumerate}
Using this theorem, we can calculate the derivative price at time $t$ from its final payoff at time $T$, by taking expectation
\[
V_t=\expec^\mathbb{Q}[e^{-r(T-t)}V_T\mid \mathcal{F}_t]
\]
where one can write $V_T=V_t\cdot f(T-t)$ to get rid of conditional expectation, and take integral eventually after finding out the upper/lower bound of the integral.
\begin{theorem}[Black's Formula]
\hfill\\\normalfont Let $F_t$ be the forward contract on risky asset with maturity $T'>0$. By forward price formula, we have
\[
F_t=S_te^{r(T'-t)}=S_0 e^{rT'}e^{-\frac{1}{2}\sigma^2t+\sigma\tilde{W}_t}=F_0e^{-\frac{1}{2}\sigma^2t+\sigma\tilde{W}_t}
\]
Here,$F_T$ is a martingale under $\mathbb{Q}$ since $e^{-\frac{1}{2}\sigma^2t+\sigma\tilde{W}_t}$ is a $L_T$ term in Girsanov Theorem.\\
Consider the European call option on this forward contract with option's maturity $T\in(0,T']$ and stike price $K>0$. Then the corresponding payoff at maturity $T$ is $G:=(F_T-K)^{+}$. and its price at time $0$
\[
p_0(G)=\expec^{\mathbb{Q}}[e^{-rT}(F_T-K)^{+}]
\]
Recognize that $e^{rT}p_0(G)$ corresponds to the BS formula with $0$ interest rate. Hence, we can use BS formula directly:
\[
e^{rT}p_0(G)=F_0N(d_{+})-KN(d_{-})
\]
with $d_{\pm}=\frac{1}{\sigma\sqrt{T}}[\ln(F_0/K)\pm\frac{1}{2}\sigma^2T]$
\end{theorem}
\begin{theorem}[Dividend Paying Asset]
\hfill\\\normalfont In order to stay self-financing, dividend needs to be used for reinvestment of the stocks. Therefore, we have $S_t^{(q)}=S_te^{qt}$ for the position of stockholder. Then we use the same idea to reduce the problem to a special case of BS equation.\\
By no-arbitrage, we have, under $\mathbb{Q}$ $S_t^{(q)}$ is an martingale, so
\[
S_t^{(q)}=S_0^{(q)}e^{(r-\sigma^2/2)t+\sigma\tilde{W}_t}=S_0e^{(r-\sigma^2/2)t+\sigma\tilde{W}_t}
\]
Applying Ito on above equation, we can show that
\[
S_t=S_0e^{(r-q-\sigma^2/2)t+\sigma\tilde{W}_t}
\]
and now
\begin{align*}
\expec^{\mathbb{Q}}[e^{-rT}(S_T-K)^{+}]&=e^{-qT}\expec^{\mathbb{Q}}[e^{-(r-q)T}(S_T-K)^{+}] \\
&= e^{-qT}[S_0N(d_1)-\tilde{K}^{(q)}N(d_2)]
\end{align*}
where $d_{1,2}=\frac{1}{\sigma\sqrt{T}}p\ln(S_0/\tilde{K}^{(q)})\pm\frac{1}{2}\sigma^2T]$ and $\tilde{K}^{(q)}:=Ke^{-(r-q)T}$.
\end{theorem}
In general, we have in the case of dividend paying asset
\[
c_t=S_te^{-q(T-t)}N(d_1)-Ke^{-r(T-t)}N(d_2)
\]
and
\[
p_t=Ke^{-r(T-t)}N(-d_2)-S_te^{-q(T-t)}N(-d_1)
\]
where \[
d_{1,2}=\frac{1}{\sigma\sqrt{T-t}}[\ln(S_t/K)+(r-q\pm\frac{1}{2}\sigma^2)(T-t)]
\]

\section{Change of Numeraire}
\begin{theorem}[Change of Numeraire Formula]
\hfill\\\normalfont Let $N_t$ be a numeraire. There exists a measure $\mathbb{Q}^N$ such that every European-style derivative maturing on the date $T$, we have the pricing formula
\[
\frac{V_t}{N_t}=\expec_t^{\mathbb{Q}^N}[\frac{V_T}{N_T}]
\]
Furthermore, we define $\mathbb{Q}^N$ by 
\[
L_T=\frac{\diff \mathbb{Q}^N}{\diff\mathbb{Q}}=\frac{N_T}{M_T}\cdot\frac{M_0}{N_0}
\]
\end{theorem}
\subsection{Introduction to American Options}
The distinctive feature of American options is the \textbf{early exercise privilege}, where the holder of the option can execise it at \textbf{any time} prior to the option's expiration date.\\
If exercised at time $t<T$,
\begin{itemize}
  \item American call option has payoff $(S_t-K)^{+}$
  \item American put option has payoff $(K-S_t)^{+}$.
\end{itemize}
In general,
\begin{definition}[American-style Instrument]
\hfill\\\normalfont An American-style instrument with maturity date $T$ and \textbf{payoff function} $\Lambda(S,t)$ is an option that can be exercised at any time before $T$, and its payoff if exercised at time $t<T$ is given by
\[
\Lambda(S,t)
\] 
where $S$ is the underlying asset price at time $t$. If the option is not exercised at all, then its payoff at maturity is $\Lambda(S,T)$.
\end{definition}
In most circumstances, the strike price $K$ and maturity $T$ are fixed. We define $C(S,t)$ to be the \textbf{price} of an American call at time $t$ with $T-t$ to maturity, strike price $K$ and underlying stock price $S$; define $P(S,t)$ to be the \textbf{price} of an American put at time $t$ with $T-t$ to maturity, strike price $K$ and underlying stock price $S$.\\
In discrete-time model, the price of American options, as well as whether to exercise early can be determined via ``backward induction''.
\begin{theorem}[Model-Free Bound]
\hfill\\\normalfont We have
\[
S\geq C(S,t)\geq c(S,t)\geq \max\{0,e^{-r(T-t)}(F(t,T)-K)\}
\]
where $F(t,T)$ is the time-$t$ forward price of the underlying stock for delivery at date $T$.\\
Similarly, 
\[
K\geq P(S,t)\geq p(S,t)\geq \max\{0,e^{-r(T-t)}(K-F(t,T))\}
\]
\end{theorem}
\begin{theorem}[American Call without Dividends]
\hfill\\\normalfont Assume that interest rates $r>0$. If underlying stock pays no dividends, then
\[
C(S,t)=c(S,t)
\]
This is proved by showing at any $t$, $C(S,t)>S_t-K$.
\end{theorem}
Similarly, we have the following result for put options:
\begin{theorem}[American Put]
\hfill\\\normalfont Assume that interest rate $r>0$. For every $t<T$, we have, regardless of dividends,
\[
p(S,t)<P(S,t)
\]
This is proved by showing a strategy that give higher payoff. One particular strategy can be to exercise at time $\min\{u,T\}$ where $u=\min\{t\geq 0: S_t\leq K-Ke^{-r(T-t)}\}$ and look at the two cases where $u<T$ or $u\geq T$.
\end{theorem}

\section{American Options}
The key questions to answer for the pricing of American options are
\begin{itemize}
  \item When to exercise
  \item At $t<T$, at what stock price should we exercise
\end{itemize}
It is known there are no analytic closed form formula for American option price.
\subsection{Optimal Exercise Boundary}
Recall, at any time $t\leq T$, American put price admits $P(S,t)\geq (K-S)^{+}$ whereas american call price admits $C(S,t)\geq (S-K)^{+}$.\\
For American put, at each time $t<T$, there exists a value $S^P_{\ast}(t)$ for the stock price such that
\begin{enumerate}
  \item If $S\leq S^P_{\ast}(t)$, then early exercise is \textit{optimal}, which gives $P(S,t)=(K-S)^{+}=K-S\geq 0$
  \item If $S>S^P_{\ast}(t)$, then immediate exercise is \textbf{not} optimal, and we have strict inequality $P(S,t)>(K-S)^{+}$.
\end{enumerate}
Similarly, for American call, at each time $t<T$, there exists a value $S^C_{\ast}(t)$ for the stock price such that
\begin{enumerate}
  \item If $S\geq S^C_{\ast}(t)$, then early exercise is \textit{optimal}, which gives $C(S,t)=(S-K)^{+}=S-K\geq 0$
  \item If $S<S^C_{\ast}(t)$, then immediate exercise is \textbf{not} optimal, and we have strict inequality $C(S,t)>(S-K)^{+}$.
\end{enumerate}
Therefore, we denote $S_\ast^P(t)$ the \textbf{optimal exercise boundary} for American put and $S_\ast^C(t)$ the \textbf{optimal exercise boundary} for American call. We ignore the superscript if type of option is not known.
\begin{theorem}
\hfill\\\normalfont\begin{enumerate}
\item $S_\ast^P(t)$ is an \textbf{increasing} function of $t$ on $[0,T]$.
\item $S_\ast^C(t)$ is an \textbf{decreasing} function of $t$ on $[0,T]$.
\end{enumerate}
\end{theorem}
We can use these boundaries to define exercise region $E$ and holding region $H$.\\
For American put, $E=\{(S,t)\in D: S\leq S_\ast^P(t)\}$, $H=\{(S,t)\in D: S>S_\ast^P(t)\}$.\\
For American call, $E=\{(S,t)\in D: S\geq S_\ast^C(t)\}$, $H=\{(S,t)\in D: S<S_\ast^C(t)\}$.
\begin{theorem}[Smooth Pasting Condition]
\hfill\\\normalfont For American put, we have
\[
\frac{\partial P}{\partial S}=-1\text{ at }S=S^P_\ast(t)
\]
For American call, we have
\[
\frac{\partial C}{\partial S}=1\text{ at }S=S^C_\ast(t)
\]
\end{theorem}
\subsection{Pricing Formulation of American Options}
Using delta-hedging, we note that, for American put options, when $(S,t)\in H$, Black Schole's PDE holds, i.e.
\[
\mathcal{L}_{BS}=\frac{\partial V}{\partial t}+r\frac{\partial V}{\partial S}S_t +\frac{1}{2}\sigma^2S_t^2\frac{\partial^2 V}{\partial S^2}-rV=0
\]
Otherwise, if $(S,t)\in E$, then $\mathcal{L}_{BS}<0$.(since $\diff\Pi_t<r\Pi_t\diff t$)\\
Specifically, for American put, we also have
\[
P(S,t)-(K-S)\geq 0
\]
with equality if and only if $(S,t)\in E$. Therefore, for American put, we have
\[
\mathcal{L}_{BS}(P)\cdot (P-(K-S))=0
\]
where $P(S,T)=(K,S)^{+}$, and $D=\{(S,t): S>0, 0\leq t<T\}$.\\
In general, 
\begin{theorem}[Linear Complementarity Problem]
\hfill\\\normalfont Suppose $q$ is the dividend yield of the underlying asset, and let the payoff function be
\[
\psi(S)=\begin{cases}
S-K & \text{ (American call)}\\
K_S & \text{ (American put)}
\end{cases}
\]
LCP is stated as below:
\[
\min\{-\frac{\partial V}{\partial t}-\frac{1}{2}\sigma^2S^2\frac{\partial^2V}{\partial S^2}-(r-q)S\frac{\partial V}{\partial S}+rV, V-\psi\}=0
\]
and 
\[
V(S,T)=\psi^{+}
\]
where $D=\{(S,t):S>0, 0\leq t<T\}$.
\end{theorem}
The LCP requires us to solve for $V(S,t)$ and $S_{\ast}(t)$ simultaneously.\\
Take American put option as an example, we have
\begin{enumerate}
  \item If $S>S^{P}_{\ast}(t)$, then $\mathcal{L}_{BS}(P)=0$.
  \item $P(S_{\ast}^P(t), t)=K-S_{\ast}^P(t)$.
  \item $\frac{\partial P}{\partial S}(S_{\ast}^P(t), t)=-1$.
  \item $P(S,T)=(K-S)^{+}$>
\end{enumerate}
American call case is similar.
\subsection{Optimal Stopping Time}
Equivalently, we want to solve what is the optimal stopping time.
\begin{definition}[Stopping Time] 
\hfill\\\normalfont A \textbf{stopping time} is a random variable $\tau:\Omega\to\mathbb{R}^{+}$ such that
\[
\{\tau>t\}\in\mathcal{F}_t, t>0
\]
\end{definition}
\begin{definition}[Hitting Times]
\hfill\\\normalfont The hitting time of level $x$ by stochastic process $X_t$ is defined as
\[
\tau_x=\inf\{t\in\mathbb{R}^{+}: X_t=x\}
\]
\end{definition}
Define
\[
P(S,t)=\sup_{t\leq \tau\leq T}\expec_t^{\mathbb{Q}}[e^{-r(\tau-t)}(K-S_{\tau})^{+}]
\]
where the supremum is taken over all possible stopping times, then the above supremum is reached at the optimal stopping time $\tau_{opt}$ such that
\[
\tau_{opt}=\inf_u\{t\leq u\leq T: P(S_u, u)=K-S_u\}
\]
We recognize $\tau_{opt}$ is the hitting time of the optimal exercise boundary $S_\ast(t)$, since
\begin{itemize}
  \item Before hitting boundary, $e^{-rt}P(S_t,t)$ is martingale, and
  \item after hitting, $e^{-rt}P(S_t,t)$ is a supermartingale
\end{itemize}
This is because, by Ito's lemma twice,
\begin{align*}
\diff (e^{-rt}P(S_t,t))&=e^{-rt}(\frac{\partial P}{\partial t}+\frac{1}{2}\sigma^2 S_t^2\frac{\partial^2 P}{\partial S^2}+rS_t\frac{\partial P}{\partial S}-rP)\diff t\\
&+e^{-rt}\sigma S_t\frac{\partial P}{\partial S}\diff W_t
\end{align*}
\subsection{Perpetual American Put Options}
Perpetual American put options works like usual American put options, except hat there is no maturity date. We recognize the price of perpetual American option $P_\infty(S)$, is indepedent of time $t$.
\begin{theorem}[Price of Perpetual American Put]
\hfill\\\normalfont 
\[
P_\infty(S)=(K-S_{\ast})(\frac{S}{S_{\ast}})^{\mu_{-}}
\]
Here, $S_{\ast}=\frac{\mu_{-}}{\mu_{-}-1}K$, $\mu_{\pm}=\frac{-(r-\sigma^2/2)\pm\sqrt{(r-\sigma^2/2)^2+2r\sigma^2}}{\sigma^2}$.
\end{theorem}
To derive, use $\frac{\partial P}{\partial t}=0$ to turn BS PDE into 2nd order PDE. The additional conditions are
\begin{itemize}
  \item $P_{\infty}(S_{\ast})=K-S_{\ast}$
  \item $\frac{\partial P_{\infty}}{\partial S}(S_{\ast})=-1$
  \item $P_{\infty}(S)\to 0$ as $S\to \infty$.
\end{itemize}

\section{Barrier Option}
Barrier options has the payoff the European option subject to whether a prescribed barrier $B$ has been hit. There are two types of conditions:
\begin{itemize}
  \item Options are activiated upon hitting the barrier(knock-in)
  \item Options are disactivated upon hitting the barrier(knock-out)
\end{itemize}
Also, there are two types of barrier:
\begin{itemize}
  \item Downstream barrier, where $B<S$, the current stock price
  \item Upstream barrier, where $B>S$
\end{itemize}
We pay particular attention in studying the \textbf{down-and-out} call, which pays, at maturity $T$
\[
\begin{cases}
(S_T-K)^{+} & \text{ if the lower barrier is never hit}\\
0 & \text{ otherwise}
\end{cases}
\]
We denote the price of the down-and-out call with time to maturity $T$ by $c_{do}(S,B,K,T)$, where $S$ is the underlying stock price today at time $0$, $B$ the downstream barrier and $K$ the strike price.\\
The payoff can be written as such:
\[
c_{do}(S_T,B,K,0)=(S_T-K)\mathbf{1}_F
\]
where $F=\{S_T\geq K, \min_{0\leq t\leq T} S_t>B\}$.
\begin{theorem}[In-out parity]
\hfill\\\normalfont Knock-out option + knock-in option = vanilla option.
\end{theorem}
\subsection{PDE Formulation}
Prior to knock-out, the option is alive and satisfies BS PDE:
\[
\frac{\partial V}{\partial t}+r\frac{\partial V}{\partial S}S_t +\frac{1}{2}\sigma^2S_t^2\frac{\partial^2 V}{\partial S^2}=rV
\]
The conditions imposed by barrier options enter through \textbf{boundary conditions} and \textbf{solution domains}:
\begin{enumerate}
  \item When barrier is hit, option becomes worthless: $V(B,t)=0$ for all $t\in [0,T]$.
  \item Solution domain for downstream barrier is: $\{B<S<\infty\}\times[0,T)$.
  \item Terminal condition is 
  \[
V(S,T)=\begin{cases} (S-K)^{+} &\text{ for call}\\
(K-S)^{+} &\text{ for put}
\end{cases}
  \]
\end{enumerate}
\subsection{Martingale Pricing}
By risk-neutral valuation, we have
\[
c_{do}(S,B,K,T)=e^{-rT}\expec^{\mathbb{Q}}[(S_T-K)\mathbf{1}_F]
\]
where $F=\{S_T\geq K, \min_{0\leq t\leq T} S_t>B\}$. Let $V_t^{(1)}=S_t\mathbf{1}_F$ and $V_t^{(2)}=K\mathbf{1}_F$, then by splitting the payoff and observe $\frac{V_T}{M_T}$ is a martingale under $\mathbb{Q}$, we have
\[
c_{do}(S,B,K,T)=V_0^{(1)}-V_0^{(2)}
\]
We can then evaluate $V_0^{(1)}$ under $\mathbb{Q}^S$ and $V_0^{(2)}$ under $\mathbb{Q}$.\\
By change of numeraire, we have
\[
\frac{V_0^{(1)}}{S_0}=\expec^{\mathbb{Q}^S}[\frac{V_T^{(1)}}{S_T}]\;\;\;\;\;\;\frac{V_0^{(1)}}{M_0}=\expec^{\mathbb{Q}}[\frac{V_T^{(1)}}{M_T}]
\]
Therefore, $V_0^{(1)}=S_0\mathbb{Q}^S(F)$ and $V_0^{(2)}=Ke^{-rT}\mathbb{Q}(F)$.\\
The problem then reduces to find the two probabilities. Note, the stock price has teh following dynamic:
\begin{itemize}
  \item Under $\mathbb{Q}^S$: $\frac{\diff S_t}{S_t}=(r+\sigma^2)\diff t+\sigma \diff W_t^S$
  \item Under $\mathbb{Q}$: $\frac{\diff S_t}{S_t}=r\diff t+\sigma\diff W_t$
\end{itemize}
Equivalently, we observe the log-prices is of the form, under $\mathbb{P}$:
\[
\diff (\ln S_t)=\mu\diff t+\sigma\diff W_t
\]
where $\mu=\begin{cases} r+\frac{\sigma^2}{2}&\text{ if }\mathbb{P}=\mathbb{Q}^S\\r-\frac{\sigma^2}{2}&\text{ if }\mathbb{P}=\mathbb{Q}\end{cases}$.\\
We transform the $F$'s representation using log prices: $F=\{\ln S_T-\ln S_0\geq \ln K-\ln S_0, \min_{0\leq t\leq T}(\ln S_t-\ln S_0)>\ln B-\ln S_0\}$. Let us denote 
\begin{itemize}
\item $X_t:=\ln S_t-\ln S_0$
\item $m_T=\min_{0\leq t\leq T} X_t$
\item $x=\ln K-\ln S_0$.
\item $m=\ln B-\ln S_0$.
\end{itemize}
Then $F=\{X_T\geq x, m_T>m\}$, and $X_t=\mu t+\sigma W_t\sim N(\mu t, \sigma^2t)$ is the Brownian motion with drift under $\mathbb{P}$, some probablistic measure.\\
For our case of down-and-out options, we have $m<0$, where $x$ can be larger or smaller than $m$.\\
instead of computing $\mathbb{P}(F)$, we study the special case $\mathbb{P}(A)$, defined by
\[
A=\{W_T\geq x, m_T\leq m\}\text{ with restriction } \underbrace{m\leq 0}_{\text{auto. satisfied}}, \underbrace{m\leq x}_{\text{assumption}}
\]
first.
By using the reflection principle 
\[
\expec^{\mathbb{P}}[\mathbf{1}_Ag(W_T)]=\expec^{\mathbb{P}}[\mathbf{1}_Bg(2m-W_T)]
\]
and a change of measure, we can make $X_T=\sigma \tilde{W}_T$ a brownian motion without drift to simplify computation, where $\tilde{W}_T=W_T+\frac{\mu}{\sigma}T$ is a brownian motion under $\tilde{\mathbb{P}}$. Specifically, the Radon Nikodym derivative is given by
\[
\frac{\diff \mathbb{P}}{\diff \tilde{\mathbb{P}}}=\exp\{-\frac{\mu^2}{2\sigma^2}T+\frac{\mu}{\sigma}\tilde{W}_T\}:=g(\tilde{W}_T)
\]
Then $A=\{\tilde{W}_T\geq \frac{x}{\sigma}, \tilde{m}_T\leq \frac{m}{\sigma}\}$ and $B=\{\tilde{W}_T\leq \frac{2m}{\sigma}-\frac{x}{\sigma}\}$, where $\frac{m}{\sigma}\leq \frac{x}{\sigma}, \frac{m}{\sigma}\leq 0$.\\
Direct computation yields $\mathbb{P}(A)=\expec^{\tilde{P}}[\mathbf{1}_B\cdot g(\frac{2m}{\sigma}-\tilde{W}_T)] = e^{\frac{2\mu m}{\sigma^2}N(\frac{2m-x+\mu T}{\sigma\sqrt{T}})}$. 
\begin{theorem}
\hfill\\\normalfont For $m\leq 0$ and $m\leq x$,
\begin{itemize}
  \item $\mathbb{P}(X_T\geq x, m_T\leq m)=e^{\frac{2\mu m}{\sigma^2}}N(\frac{2m-x+\mu T}{\sigma\sqrt{T}})$
  \item $\mathbb{P}(X_T\geq x, m_T> m)=N(\frac{\mu T-x}{\sigma\sqrt{T}})-e^{\frac{2\mu m}{\sigma^2}}N(\frac{2m-x+\mu T}{\sigma\sqrt{T}})$
\end{itemize}
\end{theorem}
Suppose in a particular case $x=m$, then event $\{X_T\geq m, m_T>m\}=\{m_T>m\}$. 
we have, for $m<0$,
\begin{itemize}
  \item $\mathbb{P}(m_T>m)=N(\frac{\mu T-m}{\sigma\sqrt{T}})-e^{\frac{2\mu m}{\sigma^2}}N(\frac{m+\mu T}{\sigma\sqrt{T}})$
  \item $\mathbb{P}(m_T<m)=N(\frac{m-\mu T}{\sigma\sqrt{T}})+e^{\frac{2\mu m}{\sigma^2}}N(\frac{m+\mu T}{\sigma\sqrt{T}})$
\end{itemize}
In general, define the minimum over the time window $[t,T]$ by $m_{[t,T]}=\min_{t\leq u\leq T}X_u$, where $d\iff X_u=\mu\diff u+\sigma \diff W_u$ with $W_u$ a $\mathbb{P}$ Brownian motion. For $m\leq 0$,
\begin{itemize}
  \item $\mathbb{P}(m_{[t,T]}>m)=N(\frac{\mu\tau-m}{\sigma\sqrt{\tau}})-e^{\frac{2\mu m}{\sigma^2}}N(\frac{m+\mu\tau}{\sigma\sqrt{\tau}})$
  \item $\mathbb{P}(m_{[t,T]}>m)=N(\frac{m-\mu\tau}{\sigma\sqrt{\tau}})+e^{\frac{2\mu m}{\sigma^2}}N(\frac{m+\mu\tau}{\sigma\sqrt{\tau}})$
\end{itemize}
where $\tau=T-t$.\\
Similarly, define the maximum over the time window $[t,T]$ by $M_{[t,T]}=\max_{t\leq u\leq T}X_u$. For $M\geq 0$,
\begin{itemize}
  \item $\mathbb{P}(M_{[t,T]}\geq M)=N(\frac{\mu\tau-M}{\sigma\sqrt{\tau}})+e^{\frac{2\mu M}{\sigma^2}}N(\frac{-M-\mu\tau}{\sigma\sqrt{\tau}})$
  \item $\mathbb{P}(M_{[t,T]}\geq M)=N(\frac{M-\mu\tau}{\sigma\sqrt{\tau}})-e^{\frac{2\mu M}{\sigma^2}}N(\frac{-M-\mu\tau}{\sigma\sqrt{\tau}})$
\end{itemize}
\subsection{Barrier Option Price}
We collect the following parameters $d_1,\ldots, d_8$. They exist in pairs.
\begin{itemize}
  \item $d_1=\frac{\ln\frac{S}{K}+(r-q+\frac{\sigma^2}{2})T}{\sigma\sqrt{T}}$
  \item $d_2=d_1-\sigma\sqrt{T}$.
  \item $d_3=\frac{\ln \frac{B^2}{SK}+(r-q+\frac{\sigma^2}{2})T}{\sigma\sqrt{T}}$
  \item $d_4=d_3-\sigma\sqrt{T}$.
  \item $d_5=\frac{\ln \frac{S}{B}+(r-q+\frac{\sigma^2}{2})T}{\sigma\sqrt{T}}$
  \item $d_6=d_5-\sigma\sqrt{T}$.
  \item $d_7=\frac{\ln \frac{B}{S}+(r-q+\frac{\sigma^2}{2})T}{\sigma\sqrt{T}}$
  \item $d_8=d_7-\sigma\sqrt{T}$.
\end{itemize}
To price the barrier option where $B<K$, 
\[
c_{do}(S,B,K,T)=S_0\mathbb{Q}^S(F)-Ke^{-rT}\mathbb{Q}(F)
\]
where $S_0\mathbb{Q}^S(F)=S_0N(d_1)-S_0(\frac{B}{S_0})^{1+\frac{2r}{\sigma^2}}N(d_3)$, and \\$-Ke^{-rT}\mathbb{Q}(F)=-Ke^{-rT}N(d_2)+Ke^{-rT}(\frac{B}{S_0})^{\frac{2r}{\sigma^2}-1}N(d_4)$.\\Equivalently, 
\[
c_{do}(S,B,K,T)=c(S,K,T)-(\frac{B}{S})^{\frac{2r}{\sigma^2}-1}c(\frac{B^2}{S},K,T)
\]
where $c(S,K,T)$ is teh price function of the usual European call option with strike $K$, spot price $S$ and time to maturity $T$.
For case $B\geq K$,
\begin{align*}
c_{do}(S,B,K,T)&=SN(d_5)-Ke^{-rT}N(d_6)-(\frac{B}{S})^{1+\frac{2r}{\sigma^2}}SN(d_7)\\
&+(\frac{B}{S})^{\frac{2r}{\sigma^2}-1}Ke^{-rT}N(d_8)
\end{align*}

\section{Asian Options}
Asian options are options whose payoff depends on some form of averaging of underlying asset price. The average will be taken over $[0,T]$. Here, we only consider \textbf{European style} options.
The options can be 
\begin{itemize}
  \item Fixed Strike, where $(A_T-K)^{+}$ is the payoff of call option, and $(K-A_T)^{+}$ of put.
  \item Floating Strike, where $(S_T-A_T)^{+}$ is the payoff of call option, and $(A_T-S_T)^{+}$ for put.
\end{itemize}
The type of averaging can be
\begin{itemize}
  \item $\frac{1}{n}\sum_{i=1}^n S_{t_i}$ for discretely sampled arithmetic
  \item $\frac{1}{T}\int_0^T S_t\diff t$ for continuously sampled arithemetic
  \item $\exp(\frac{1}{n}\sum_{i=1}^n \ln S_{t_i})$ for discretely sampled geometric
  \item $\exp(\frac{1}{n}\int_0^T \ln S_t\diff t)$ for continuously sampled geometric
\end{itemize}
We only consider continuously sampled Asian options.
\subsection{Multivariate Ito's Lemma}
Consider two Ito process evolving over time:
\begin{itemize}
  \item $\diff X_t=a(X_t,Y_t)\diff t+b(X_t,Y_t)\diff W_t^X$
  \item $\diff Y_t=c(X_t,Y_t)\diff t+d(X_t,Y_t)\diff W_t^Y$
  \item $\rho=\text{Corr}[\diff W_t^X, \diff W_t^Y]$ where $\rho$ is the correlation between $\diff W^X$ and $\diff W^Y$.
\end{itemize}
Consider another stochastic process $V_t:=V(X_t,Y_t)$. Then we have
\begin{align*}
\diff V_t &= \frac{\partial V}{\partial X}(X_t,Y_t)\diff X_t+ \frac{\partial V}{\partial Y}(X_t,Y_t)\diff Y_t\\
&+\frac{1}{2}\frac{\partial^2 V}{\partial X^2}(X_t,Y_t)(\diff X_t)^2+\frac{1}{2}\frac{\partial^2 V}{\partial Y^2}(X_t,Y_t)(\diff Y_t)^2\\
&+\frac{\partial^2 V}{\partial X\partial Y}(X_t,Y_t)\diff X_t\diff Y_t
\end{align*}
Substituting $\diff X_t$ and $\diff Y_t$, we will have the multivariate Ito's lemma. When $V$ is also a function of $t$, then we need to add an extra then $\frac{\partial V}{\partial t}\diff t$ into the above equation.
\subsection{General PDE Framework}
The price function $V(S_t,I_t,t)$ for \textbf{parth-dependent} options depends on $S_t, I_t, t$, where $I_t$ is the \textbf{path-dependent variable}
\[
I_t=\int_0^t f(S_u,u)\diff u
\]
Equivalently, $\diff I_t=f(S_t,t)\diff t$.\\
As usual, we assume under $\mathbb{Q}$, stock price follows $\diff S_t=rS_t\diff t+\sigma S_t\diff W_t$.\\
Applying multivaraite Ito Lemma to $V_t$, we have
\[
\diff V_t=(\frac{\partial V}{\partial t}+\frac{1}{2}\sigma^2 S_t^2\frac{\partial^2 V}{\partial S^2})\diff t + \frac{\partial V}{\partial S}\diff S_t+\frac{\partial V}{\partial I}\diff I_t
\]
since $(\diff I_t)^2=0$ and $\diff S_t\diff I_t=0$.\\
Using same argument of delta hedging, we arrive at
\[
\frac{\partial V}{\partial t}+\frac{1}{2}\sigma^2 S^2\frac{\partial^2 V}{\partial S^2}+rS\frac{\partial V}{\partial S}+f(S,t)\frac{\partial V}{\partial I}-rV=0
\]
Set the continuously sampled arithmetic mean as 
\[
A_T=\frac{1}{T}\int_0^T S_u\diff u
\]
We want to obtain $f(S_t,t)$ that contains information about $A_T$.
\begin{itemize}
  \item $f(S_t,t)=S_t$.\\Then $I_t=\int_0^t S_u\diff u$, and $\diff I_t=S_t\diff t$. This allows us to relate $A_T=\frac{I_T}{T}$.\\
  The PDE is
  \[
\frac{\partial V}{\partial t}+\frac{1}{2}\sigma^2 S^2\frac{\partial^2 V}{\partial S^2}+rS\frac{\partial V}{\partial S}+S\frac{\partial V}{\partial I}-rV=0
  \]
  The terminal condition is 
  \[
V(S,I,T)=\begin{cases}
(I/T-S)^{+} & \text{ floating put}\\
(S-I/T)^{+} & \text{ floating call}\\
(I/T-K)^{+} & \text{ fixed strike call}\\
(K-I/T)^{+} & \text{ fixed strke put}
\end{cases}
  \]
  \item $f(S_t,t)=\frac{1}{t}(S_t-A_t)$ where $A_t=\frac{1}{t}\int_0^t S_u\diff u$. By Ito, $\diff A_t=\frac{1}{t}(S_t-A_t)\diff t$, or equivalently, $A_t=\int_0^t\frac{1}{u}(S_u-A_u)\diff t+S_0$.\\
  The PDE will be of the form
  \[
\frac{\partial V}{\partial t}+\frac{1}{2}\sigma^2 S^2\frac{\partial^2 V}{\partial S^2}+rS\frac{\partial V}{\partial S}+\frac{S_A}{t}\frac{\partial V}{\partial A}-rV=0
  \]
  and the terminal condition is
  \[
V(S,A,T)=\begin{cases}
(A-S)^{+} & \text{ floating put}\\
(S-A)^{+} & \text{ floating call}\\
(A-K)^{+} & \text{ fixed strike call}\\
(K-A)^{+} & \text{ fixed strke put}
\end{cases}
  \]
\end{itemize}
\subsection{Roger-Shi Method}
We investigate the second approach. We define the new variable
\[
X_t=\frac{K-\frac{1}{T}\int_0^t S_u\diff u}{S_t}=\frac{K-\frac{I_t}{T}}{S_t}
\]
with $X_0=\frac{K}{S_0}$.\\
This allows $V(S,I,t)=e^{-r(T-t)}\expec_t^{\mathbb{Q}}[(A_T-K)^{+}]$.\\
Note, 
\[
\expec_t^{\mathbb{Q}}[(A_T-K)^{+}]=S_t\expec_t^{\mathbb{Q}}[(\frac{1}{T}\int_t^T \frac{S_u}{S_t}\diff u-X_t)^{+}]
\]
Define $h(X,t)=\expec_t^{\mathbb{Q}}[(\frac{1}{T}\int_t^T \frac{S_u}{S_t}\diff u-X)^{+}]$ and let 
\[
H(X,t)=e^{-r(T-t)}h(X,t)
\]
Then we have
\[
V(S,I,t)=SH(X,t)
\]
The resulting PDE with respect to $H$ is
\[
\frac{\partial H}{\partial t}+\frac{1}{2}\sigma^2 X^2\frac{\partial^2 H}{\partial X^2}-(\frac{1}{T}+rX)\frac{\partial H}{\partial X}=0
\]
where $-\infty<X\infty$ and $t\in[0,T)$, with terminatl condition $V(S,I,T)=(\frac{I}{T}-K)^{+}$, or equivalently $H(X,T)=(-X)^{+}$.\\

The closed form analytic solution exists to $H(X,t)$ if $X\leq 0$. This is because the terminal condition is simplied to $H(X,T)=-X\geq 0$, a linear equation.\\
Analytic solution of $H(X,t)$ is
\[
H(X,t)=\frac{1-e^{-r(T-t)}}{rT}-e^{-r(T-t)}X
\]
and 
\[
V(S,I,t)=(\frac{I}{T}-K)e^{-r(T-t)}+\frac{1-e^{-r(T-t)}}{rT}S
\]
However, there is no closed form analytic solution for $X>0$. One need to solve this equation:
\[
\frac{\partial H}{\partial t}+\frac{1}{2}\sigma^2 X^2\frac{\partial^2 H}{\partial X^2}-(\frac{1}{T}+rX)\frac{\partial H}{\partial X}=0
\]
with domain $t\in[0,T), x\in(0,\infty)$ and terminal condition $H(0,t)=\frac{1-e^{-r(T-t)}}{rT}$, $H(\infty, t)=0$ and $H(X,T)=-, X\in (0,\infty)$.
\subsection{Put call Parity for Fixed Strike Arithmetic Asian Option}
The put call parity is
\begin{align*}
c_\text{fix}(S_t,I_t, T)&-p_\text{fix}(S_t,I_t,T)\\
&=(\frac{I_t}{T}-K)e^{-r(T-t)}+\frac{1-e^{-r(T-t)}}{rT}S_t
\end{align*}
\subsection{Geometric Asian option}
Geometric Asian option in continuous manner has the running geometric $G_t=\exp(\frac{1}{t}\int_0^t \ln S_u\diff u)$. By Ito lemma,
\[
\diff G_t=\frac{G_t\ln \frac{S_t}{G_t}}{t}\diff t
\]
The PDE is 
\[
\frac{\partial V}{\partial t}+\frac{1}{2}\sigma^2 S^2\frac{\partial^2 V}{\partial S^2}+rS\frac{\partial V}{\partial S}+\frac{G\ln \frac{S}{G}}{t}\frac{\partial V}{\partial G}-rV=0
\]
The solution domain is $\{S>0, G>0, t\in [0,T)\}$.\\
Let's try to solve for the fixed strike geometric Asian. Denote the time-$t$ price by $c_\text{fix}(S,G,t)$ where $0\leq t\leq T$. Risk neutral valuation asserts
\[
c_\text{fix}(S,G,t)=e^{-r(T-t)}\expec_t^{\mathbb{Q}}[(G_T-K)^{+}]
\]
For $t\leq T$, define $X_t=\frac{1}{t}\int_0^t\ln S_u\diff u$. This gives $G_t=e^{X_t}$, and the payoff is $(e^{X_T}-K)^{+}$.\\
By studying the distribution of $\ln S_u$, we have
\begin{align*}
&X_T\\
&=\frac{t}{T}X_t+\frac{1}{T}\int_t^T(\ln S_t+(r-\frac{\sigma^2}{2})(u-t)+\sigma(W_u-W_t))\diff u\\
&=\frac{t}{T}X_t+\frac{T-t}{T}\ln S_t\\
&+(r-\sigma^2/2)\frac{(T-t)^2}{2T}+\frac{\sigma}{T}(\int_t^T (W_u-W_t)\diff u)
\end{align*}
Note, the last term is normally distributed with mean $0$ and variance $\frac{\sigma^2}{T^2}\frac{(T-t)^3}{3}$.\\
By setting
\begin{itemize}
  \item $\bar{\mu}=(r-\sigma^2/2)\frac{(T-t)^2}{2T}$
  \item $\bar{\sigma}=\frac{\sigma}{T}\sqrt{\frac{(T-t)^3}{3}}$
\end{itemize}
Therefore, we have
\[
G_T=G_t^{t/T}S_t^{(T-t)/T}\exp(\bar{\mu}+\bar{\sigma}\phi)
\]
Using normal calcualtion, we have
\[
c_\text{fix}(S,G,t)=e^{-r(T-t)}(G_t^{t/T}S_t^{(T-t)/T}e^{\bar{\mu}+\bar{\sigma}^2/2}N(d_1)-KN(d_2))
\]
where $d_2=\frac{1}{\bar{\sigma}}(\frac{t}{T}\ln G_t+\frac{T-t}{T}\ln S_t-\ln K+\bar{\mu})$ and $d_1=d_2+\bar{\sigma}$.

\section{Lookback Options \& Two Assets Options}
\subsection{Lookback Option}
Lookback options are path dependent options whose payoffs depends on the \textbf{maximum} or \textbf{minimum} of the underlying stock price attained over a certain period of time, known as \textbf{lookback period}. We set the lookback period to be $[0,T]$.\\
Let minimum value of underlying asset over the lookback period $[0,T]$ by
\[
m_T=\min_{0\leq t\leq T}S_t
\]
and the maximum value by
\[
M_T=\max_{0\leq t\leq T}S_t
\]
There are 2 types of lookback options:
\begin{enumerate}
	\item \textbf{Fixed Strike}:
	\begin{itemize}
		\item Call: $(M_T-K)^+$
		\item Put: $(K-m_T)^+$
	\end{itemize}
	\item \textbf{Floating Strike}:
	\begin{itemize}
		\item Call: $(S_T-m_T)$
		\item Put: $(M_T-S_T)$
	\end{itemize}
\end{enumerate}
Recall, the general PDE for European path-depedent options is
\[
\frac{\partial V}{\partial t}+\frac{1}{2}\sigma^2 S^2\frac{\partial^2 V}{\partial S^2}+rS\frac{\partial V}{\partial S}+f(S,t)\frac{\partial V}{\partial I}-rV=0
\]
where $I_t=\int_0^t f(S_u,u)\diff u$ is the path dependent variable.\\
\begin{theorem}\normalfont Suppose $f(x)$ continuous and positive on $[a,b]$. The maximum of $f(x)$ on $[a,b]$ can be given by $\max_{a\leq x\leq b}f(x)=\lim_{n\to\infty}(\int_a^b f(x)^n\diff x)^{\frac{1}{n}}$.\end{theorem}
\begin{theorem}\normalfont It follows that $M_t=\lim_{n\to\infty}(\int_0^t S_u^n\diff u)^{\frac{1}{n}}$ and $m_t=\lim_{n\to\infty}(\int_0^tS_u^{-n}\diff u)^{-\frac{1}{n}}$.\end{theorem}
Define $I_{n,t}=\int_0^tS_u^n\diff u$, and $M_{n,t}=I_{n,t}^\frac{1}{n}$. Then $M_t=\lim_{n\to\infty}M_{n,t}$. By Ito's Lemma,
\[
\diff I_{n,t}=S_t^n\diff t
\]
and
\[
\diff M_{n,t}=\frac{1}{n}\frac{S_t^n}{M_{n,t}^{n-1}}\diff t
\]
Ito's Lemma gives
\begin{align*}
\diff V_t&=\frac{\partial V}{\partial t}\diff t+\frac{\partial V}{\partial S}\diff S + \frac{\partial V}{\partial M_n}\diff M_{n,t}+\frac{1}{2}\frac{\partial^2 V}{\partial S^2}(\diff S_t)^2\\
&=(\frac{\partial V}{\partial t}+\frac{1}{2}\sigma^2S_t^2\frac{\partial^2 V}{\partial S^2}+\frac{1}{n}\frac{S_t^n}{M_{n,t}^{n-1}}\frac{\partial V}{\partial M_n})\diff t +\frac{\partial V}{\partial S}\diff S_t
\end{align*}
Note, $(\diff(M_{n,t}))^2$ does not appear since $\diff(M_{n,t})$ is deterministic. Then by Delta hedging and self financing, we arrive at \[
\diff \Pi_t=(\frac{\partial V}{\partial t}+\frac{1}{2}\sigma^2 S_t^2\frac{\partial^2 V}{\partial S^2})\diff t
\]
The term $\frac{1}{n}\frac{S_t^n}{M_{n,t}^{n-1}}\leq \frac{S_t}{n}\to 0$ as $n\to\infty$. This gives back the Black-Scholes PDE, where
\[
\frac{\partial V}{\partial t}+r\frac{\partial V}{\partial S}S_t +\frac{1}{2}\sigma^2S_t^2\frac{\partial^2 V}{\partial S^2}=rV
\]
For American lookback options, we have $\mathcal{L}_{BS}\leq 0$ with equality if and only if $(S,t)$ is in holding region.\\
For European floating strike lookback put option, it satisfies BS PDE with following boundary and terimal conditions
\begin{enumerate}
	\item $V(S,M,T)=M-S$
	\item $V(0,M,t)=e^{-r(T-t)}M$
	\item $\frac{\partial V}{\partial M}(M,M,t)=0$
\end{enumerate}
Solution domain is $\{(S,M,t): 0<S<M, 0\leq t\leq T\}$.
\subsection{Two Asset Options}
Suppose $\diff X_t = \mu_X X_t\diff t + \sigma_XX_t\diff W_t^X$ and $\diff Y_t = \mu_Y Y_t\diff t + \sigma_YY_t\diff W_t^Y$ are two stocks with $\diff W_t^X\diff W_t^Y=\rho \diff t$. We can use Delta-hedging to derive the PDE. The key point is Multivariate Ito's Lemma which gives
\begin{align*}
\diff V_t &= (\frac{\partial V}{\partial t}+\frac{1}{2}\sigma_X^2 X_t^2\frac{\partial^2 V}{\partial X^2}+\frac{1}{2}\sigma_Y^2 Y_t^2\frac{\partial^2 V}{\partial Y^2}\\
&+\rho\sigma_X\sigma_Y X_tY_t\frac{\partial^2 V}{\partial X\partial Y})\diff t \\
&+\frac{\partial V}{\partial X}\diff X_t+ \frac{\partial V}{\partial Y}\diff Y_t
\end{align*}
Therefore, $\Pi_t=V_t-\frac{\partial V}{\partial X}X_t-\frac{\partial V}{\partial Y}Y_t$ in order to be riskless. Finally, we have
\begin{align*}
&\frac{\partial Y}{\partial t}+\frac{1}{2}\sigma_X^2 X^2\frac{\partial^2 V}{\partial X^2}+\frac{1}{2}\sigma_Y^2 Y^2\frac{\partial^2 V}{\partial Y^2}+\rho \sigma_X\sigma_Y XY\frac{\partial^2 V}{\partial X\partial Y}\\
&+rX\frac{\partial V}{\partial X}+rY\frac{\partial V}{\partial Y}-rV=0
\end{align*}
The solution domain is $\{X>0, Y>0, t\in[0,T)\}$, with terminal condition $V(X,Y,T)=f(X,Y)$.
\subsection{Exchange Options}
European Exchange Options that allows exchange of $Y_T$ for $X_T$ has the terminal payoff at maturity
\[
V(X_T,Y_T,T)=\max\{X_T-Y_T,0\}
\]
It obeys the two asset option PDE. However, to make it solvable, we let $Z=\frac{X}{Y}$ So that $V(X,Y,t)=Y\cdot H(Z,t)$, and we solve for $H$.\\
The partial derivatives are
\begin{itemize}
	\item $\frac{\partial V}{\partial t}=Y\frac{\partial H}{\partial t}$
	\item $\frac{\partial V}{\partial Y}=H-Z\frac{\partial H}{\partial Z}$
	\item $\frac{\partial^2 V}{\partial Y^2}=\frac{1}{Y}Z^2\frac{\partial^2 H}{\partial Z^2}$
	\item $\frac{\partial^2 V}{\partial X\partial Y}=-\frac{X}{Y^2}\frac{\partial^2 H}{\partial Z^2}$.
\end{itemize}
Therefore, we arrive at
\[
\frac{\partial H}{\partial t}+\frac{1}{2}\bar{\sigma}^2Z^2\frac{\partial^2 H}{\partial Z^2}=0
\]
where $\bar{\sigma}^2=\sigma_X^2+\sigma_Y^2-2\rho\sigma_X\sigma_Y$, and terminal condition for $H$ is $H(Z,T)=(Z-1)^+$. This can be solved as
\[
H(Z,t)=ZN(d_1)-N(d_2)
\]
where $d_1=\frac{\ln Z+\frac{\bar{\sigma}^2}{2}(T-t)}{\bar{\sigma}\sqrt{T-t}}$, $d_2=d_1-\bar{\sigma}\sqrt{T-t}$. Consiquently, $V(X,Y,t)=XN(d_1)-YN(d_2)$.
\subsection{Cross-currency Options}
Suppose underlying aseet $S_t$ is in foreign currency $f$ and payoff in domestic currency $d$. The payoff is affected by terminal asset price $S_T$ in foreign currency and terminal exchange rate $F_T$.\\
\begin{definition}\normalfont The exchange rate $F_t$ is the price of one unit foreign currency in US dollars(domestic).\end{definition}
The model is $\diff S_t=\mu_S S_t\diff t+\sigma_S S_t\diff W_t^S$ and $\diff F_t=\mu_F F_t\diff t+\sigma_F F_t\diff W_t^F$ with $\diff W_t^S\diff W_t^F=\rho\diff t$.\\
Let $V(S,F,t)$ define the options value in domestic currency at time $t$. We use delta-hedging:
\[
\Pi_t=V_t-\delta_F\cdot F_t-\delta_S\cdot S_t\cdot F_t
\]
Let $r_f$ be forieng risk-free rate. This can be treated as dividend yield of foriegn currency.\\
Self financing condition gives
\begin{align*}
\diff \Pi_t&=\diff V-\Delta_F\diff F-\underbrace{\Delta_F r_f F\diff t}_{\text{dividend}}\\&-\Delta_S\underbrace{(S\diff F+ F\diff S+\rho\sigma_S\sigma_F SF\diff t)}_{\diff(SF)}
\end{align*}
where $\diff V$ is given by
\begin{align*}
\diff V&=(\frac{\partial V}{\partial t}+\frac{1}{2}\sigma_S^2S^2\frac{\partial^2 V}{\partial S^2}+\frac{1}{2}\sigma_F^2 F^2\frac{\partial^2 V}{\partial F^2}\\&+\rho\sigma_S\sigma_F SF\frac{\partial^2 V}{\partial F\partial S})\diff t\\
&+\frac{\partial V}{\partial S}\diff S+\frac{\partial V}{\partial F}\diff F
\end{align*}
This gives $\Delta_S=\frac{1}{F}\frac{\partial V}{\partial S}$ and $\Delta_F=\frac{\partial V}{\partial F}-\frac{S}{F}\frac{\partial V}{\partial S}$. Furthermore, $\diff \Pi = r_d\Pi\diff t$. Simplification gives
\begin{align*}
&\frac{\partial V}{\partial t}+\frac{1}{2}\sigma_S^2S^2\frac{\partial^2 V}{\partial S^2}+\frac{1}{2}\sigma_F^2 F^2\frac{\partial^2 V}{\partial F^2}\\&+\rho\sigma_S\sigma_F SF\frac{\partial^2 V}{\partial F\partial S}+(r_d-r_f)F\frac{\partial V}{\partial F}+(r_f-\rho\sigma_S\sigma_F)S\frac{\partial V}{\partial S}\\&-r_dV=0
\end{align*}

\end{document}
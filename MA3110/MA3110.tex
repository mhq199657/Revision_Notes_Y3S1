\PassOptionsToPackage{svgnames}{xcolor}
\documentclass[12pt]{article}



\usepackage[margin=1in]{geometry}  
\usepackage{graphicx}             
\usepackage{amsmath}              
\usepackage{amsfonts}              
\usepackage{framed}               
\usepackage{amssymb}
\usepackage{array}
\usepackage{amsthm}
\usepackage[nottoc]{tocbibind}
\usepackage{bm}
\usepackage{enumitem}

 \newcommand{\im}{\mathrm{i}}
  \newcommand{\diff}{\mathrm{d}}
\setlength{\parindent}{0cm}
\setlength{\parskip}{0em}
\newcommand{\Lim}[1]{\raisebox{0.5ex}{\scalebox{0.8}{$\displaystyle \lim_{#1}\;$}}}
\newtheorem{definition}{Definition}[section]
\newtheorem{theorem}{Theorem}[section]
\newtheorem{lemma}{Lemma}[section]
\newtheorem{corollary}{Corollary}[section]
\theoremstyle{definition}
\DeclareMathOperator{\arcsec}{arcsec}
\DeclareMathOperator{\arccot}{arccot}
\DeclareMathOperator{\arccsc}{arccsc}
\setcounter{tocdepth}{1}
\begin{document}

\title{Revision notes - MA3110}
\author{Ma Hongqiang}
\maketitle
\tableofcontents

\clearpage
\section{Review}
\begin{definition}[Limit of Sequence]
\hfill\\\normalfont For a sequence $(x_n)_{n\in \mathbb{N}}$, we say $\lim_{n\to \infty} x_n = a$ if and only if
\[
\forall \epsilon >0, \exists n_0\in \mathbb{N} \text{ such that }\forall n\geq n_0, |x_n-a|\leq \epsilon
\]
\end{definition}
Similarly, we define $\lim_{n\to \infty} x_n = \infty$ if and only if
\[
\forall M>0, \exists n_0\text{ depending on }M, \text{s.t. }\forall n\geq n_0, x_n\geq M
\]
\begin{definition}[Limit Point(Subsequential Limit in \texttt{MA2108} notes)]
\hfill\\\normalfont A number $a\in[-\infty, \infty]$ is called a \textbf{limit point} of the sequence $(x_n)_{n\in \mathbb{N}}$, if there exists an increasing sequence of indices $n_1<n_2<n_3<\cdots$ such that $\lim_{i\to \infty} x_{n_i} = a$.
\end{definition}
\begin{theorem}
\hfill\\\normalfont $\lim_{n\to \infty}x_n$ does not exist in $[-\infty, \infty]$ if and only if $(x_n)_{n\in \mathbb{N}}$ has more than 1 limit point in $[-\infty, \infty]$.
\end{theorem}
\begin{definition}[Supremum and Infimum]
\hfill\\\normalfont Let $A\subset[-\infty, \infty]$. The \textbf{supremum} of $A$, denoted by $\sup A$, is defined to be the \textbf{least upper bound} of $A$.\\
Essentially, $p = \sup A$ if and only if
\begin{enumerate}
  \item $x\leq p\forall x\in A$
  \item if $x\leq u\forall x\in A$ for some $u\in [-\infty, \infty]$, then $p\leq u$.
\end{enumerate}
The infimum is defined in a similar fashion. For detailed definition, check MA2108 revision notes.
\end{definition}
\begin{definition}[Limit Supremum and Infimum]
\hfill\\\normalfont Given a sequence of real numbers $(x_n)_{n\in \mathbb{N}}$,
\[
\lim\sup_{n\to\infty} x_n :=\lim_{n\to \infty}(\sup_{m\geq n} x_m)
\]
and 
\[
\lim\inf_{n\to\infty} x_n :=\lim_{n\to \infty}(\inf_{m\geq n} x_m)
\]
\end{definition}
\begin{theorem}
\hfill\\\normalfont $\lim\sup_{n\to \infty} x_n$ is a limit point and the \textbf{largest limit point} of the sequence $(x_n)_{n\in \mathbb{N}}$. $\lim\inf_{n\to \infty} x_n$ is the smallest limit point.
\end{theorem}
\begin{theorem}
\hfill\\\normalfont $\lim_{n\to\infty} x_n$ exists in $[-\infty,\infty]$ if and only if $\lim\sup_{n\to\infty} x_n = \lim\inf_{n\to\infty} x_n$.
\end{theorem}
\begin{definition}[Continuity]
\hfill\\\normalfont A function $f:\mathbb{R}\to\mathbb{R}$ is said to be \textbf{continuous} at $x$ if $\lim_{y\to x} f(y)$ exists and equals $f(x)$.\\
Equivalently,
\[
\forall \epsilon>0, \exists \delta>0\text{ such that } \sup_{y\in[x-\delta, x+\delta]}|f(y)-f(x)|\leq \epsilon
\]
\end{definition}
\end{document}

